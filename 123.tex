\documentclass{homework}
\author{Joseph Siu}
\class{Course Name _}
\date{\today}
\title{Homework _}

\newcommand{\Set}[1]{\{#1\}}
\newcommand{\T}[1]{\text{#1}}
\newcommand{\Al}[3]{#1 &=#2 &\text{#3}&&\\}

% Symbols
\newcommand*{\eg}{\leavevmode\unskip , e. g., \ignorespaces} % for example
\newcommand*{\ie}{\leavevmode\unskip, i. e., \ignorespaces} % that is
\newcommand{\nil}{\varnothing}
\AtBeginDocument{\def\O{\cal{O}}} % Big Oh
\AtBeginDocument{\def\C{\bb{C}}} % Complex
\newcommand{\R}{\bb{R}} % Reals
\newcommand{\Q}{\bb{Q}} % Rationals
\newcommand{\Z}{\bb{Z}} % Integers
\newcommand{\N}{\bb{N}} % Naturals
\renewcommand{\P}{\bb{P}} % Primes
\newcommand{\Pset}[1]{\mathcal{P}(#1)} %power set
\newcommand{\Relate}[2]{#1\mathcal{R}#2} %relation
\newcommand{\relate}{\mathcal{R}}
\newcommand{\F}{\bb{F}} 
\newcommand{\GF}[1][2]{\bb{F}_{#1}} 
\newcommand{\modulo}[1][n]{\Z/#1\Z} 
\newcommand{\ra}{\rightarrow}
\newcommand{\Ra}{\Rightarrow}
\newcommand{\?}{\stackrel{?}{=}}
\newcommand{\is}{\equiv}
\newcommand{\al}{\alpha}
\newcommand{\ep}{\varepsilon}
\renewcommand{\phi}{\varphi}
\newcommand{\p}{\partial}
\newcommand{\injective}{\hookrightarrow}
\newcommand{\surjective}{\twoheadrightarrow}
\newcommand{\bijective}{\hookrightarrow\mathrel{\mspace{-15mu}}\rightarrow}
\newcommand{\derivative}[2][x]{\frac{\D #2}{\D #1}}
\newcommand{\ceil}[1]{\left\lceil#1\right\rceil}
\newcommand{\floor}[1]{\left\lfloor#1\right\rfloor}
\newcommand{\near}[1]{\left\lfloor#1\right\rceil}
\newcommand{\arr}[1]{\left\langle#1\right\rangle}
\newcommand{\paren}[1]{\left(#1\right)} %pair / ()
\newcommand{\brk}[1]{\left[#1\right]} %[]
\newcommand{\abs}[1]{\left|#1\right|}
\newcommand{\curl}[1]{\left\{#1\right\}} %set {}
\newcommand{\func}[3]{#1: #2 \rightarrow #3}


\theoremstyle{definition}
\newtheorem*{claim}{Claim}
\newtheorem{definition}{Definition}
\newtheorem{theorem}{Theorem}
\newtheorem{lemma}{Lemma}

\begin{document} \maketitle

1.

Denote P = lecture will be in person

Denote Q = infection rate decreases

P IMPLIES Q

2.

$y_0=x_0$ XOR $b$

$y_1=x_1$ XOR ($x_0$ AND $b$)

$c=x_1$ AND ($x_0$ AND $b$)

Generalization

$y_0=x_0$ XOR $b$

$y_i=x_i$ XOR ($x_{i-1}$ AND $x_{i-2}$ AND ... AND $x_{0}$ AND $b$) where $i\in \Z\cap[0,n]$

$c=x_n$ AND $x_{n-1}$ AND $x_{n-2}$ AND ... AND $x_{0}$ AND $b$

3.

D = logic is difficult

S = many students like logic

E = Mathematics is easy

[(D OR NOT(s)) AND (E IMPLIES NOT(D))] IMPLIES [NOT(S) IMPLIES (NOT(E) OR NOT(D))]

(E AND D) OR S OR NOT(E) OR NOT(D)

TAUTOLOGY

4.

TRUE

FALSE

TRUE

4.

floor(x,y)="0<=x-y AND x-y<1"

floor(x,y)= "(y<=x) AND ($\forall z\in\Z$).[(z<=x) IMPLIES (z<= y)]"

round(x,y)="$\forall z\in\Z$.(|y-x|<=|z-x|)"

round(x,y)="|x-y|<1/2 OR (|x-y|=1/2 AND x<y)"

5. Let P = all processes

a. $(\forall x \in P. \forall z \in P. \neg(t(x,q,z)))$ AND $\exists x\in P. w(x,q)$

b. $\forall x \in P. \exists y \in P. \neg(w(y,x)) AND t(x,x,y)$

c. $\exists x \in P. \forall y \in P. \forall z \in P. t(y,x,z) IMPLIES w(x,y)$

\end{document}