\documentclass[11pt]{article}
\usepackage{fullpage}
\usepackage{amsthm}
\usepackage{amsfonts}
\usepackage{changepage}   % for the adjustwidth environment

\setlength{\parindent}{0pt}
\setlength{\parskip}{0.3cm}
\newcommand{\nats}{\mathbb{N}}
\newcommand{\ints}{\mathbb{Z}}

\usepackage[margin=1in]{geometry}

%%% Logic
\newcommand{\nnot}{\mathrm{NOT}}
\newcommand{\aand}{\,\,\mathrm{AND}\,\,}
\newcommand{\oor}{\,\,\mathrm{OR}\,\,}
\newcommand{\iimplies}{\,\,\mathrm{IMPLIES}\,\,}
\newcommand{\xor}{\,\,\mathrm{XOR}\,\,}
\newcommand{\iif}{\,\,\mathrm{IFF}\,\,}

% Environment for writing formal proofs======================================
%
% \nl ends line and makes next line a Numbered Line
% \ul ends line and makes next line an Unnumbered Line 
% \n increases the level of indentation ("next")
% \p decreases the level of indentation ("previous")
%\firstline  to number the first line of a proof
%
% Example:
% 
% 1     let $x \in \nats$ be arbitrary
% 2         let $y = x+1 \in \nats$
% 3         $y > x$; property of $\nats$, 2
% 4      $\exists y \in \nats. (y > x)$; construction 2, 3
% 5  $\forall x \in \nats. \exists y \in \nats. (y > x)$; generalization 1,4
%

% this can be typed as follows:
%
% \begin{formal}
% \firstline          <--- to number the first line of a proof
% \n \label{gen-start} let $x \in \nats$ be arbitrary \nl  
% \n let $y = x+1 \in \nats$ \label{defy} \nl
% $y > x$; property of $\nats$, \lref{defy} \label{gt} \nl
% \p $\exists y \in \nats. (y > x)$; \label{inside} construction \lref{defy},\lref{gt} \nl
% \p $\forall x \in \nats. \exists y \in \nats. (y > x)$; generalization \lref{gen-start}, \lref{inside}
% \end{formal}
% 
% You can use \labels anywhere in the code and \lref to refer to the line
% number.  (This is done using smartref package.)
%
%to reset the line numbering counter:
%\setcounter{linenum}{0} 

\usepackage{smartref} % for referencing the line numbers
\newcounter{linenum}
\addtoreflist{linenum}
\def\codeTabSpace{\hspace*{4mm}}
\newenvironment{formal}%
{\begin{tabbing}%
\codeTabSpace \= \hspace*{20mm} \= \hspace*{20mm} \= \hspace*{20mm} \= \kill%
}%
{\end{tabbing}%
}
\newcounter{ind}
\newcommand{\n}{\addtocounter{ind}{5}\hspace*{5mm}}
\newcommand{\p}{\addtocounter{ind}{-5}\hspace*{-5mm}}
\newcommand{\nl}{\\\stepcounter{linenum}{\scriptsize \arabic{linenum}}\>\hspace*{\value{ind}mm}}
\newcommand{\ul}{\\\>\hspace*{\value{ind}mm}}
\newcommand{\bl}{\\[-1.5mm]\>\hspace*{\value{ind}mm}}
\newcommand{\firstline}{\stepcounter{linenum}{\scriptsize \arabic{linenum}}\>}
\newcommand{\lref}[1]{\linenumref{#1}} % use this to refer to a line number
% End of environment for writing formal proofs=====================================


\begin{document}
\begin{center}
{\bf \Large \bf CSC240 Winter 2024 Homework Assignment 3}\\
\end{center}

My name and student number: Joseph Siu, 1010085701\\
The list of people with whom I discussed this homework assignment:

Hrithik Parag Shah

Serif Wu

Joseph Siu 

Sanchit Manchanda

Abhi Prajapati

Sepehr Jafari

1.
Let ${\mathcal F}$ be the set of all functions from $D$ to $D$, where $D$ is a nonempty set.\\
Consider the following two predicates
with domain ${\mathcal F} \times {\mathcal F}$:
\begin{eqnarray*}
    P(f,g) &= &\exists y \in D. \forall x \in D.[ f(g(x)) \neq y] \mbox{  and }\\
    Q(f,g) &= &\exists v \in D.[  \forall u \in D. (f(u) \neq v) \oor \forall u \in D. (g(u) \neq v)].
\end{eqnarray*}

Formally prove that $\forall f \in {\mathcal F}. \forall g \in {\mathcal F}.(P(f,g) \iimplies Q(f,g))$.\\
Remember to number all lines, indent properly, and justify all your steps, including references to the appropriate line numbers, as described in Proof Outlines. Only do one step of the proof per line.

\begin{proof}
    \hfill
    \begin{formal}
        \firstline
        \n Let $f$ be an arbitrary element of $\mathcal{F}$; \label{letf}\nl

        \n Let $g$ be an arbitrary element of $\mathcal{F}$; \label{letg}\nl

        \n Assume $P(f,g)$; \label{assumep}\nl

        $\exists y\in D. \forall x\in D. (f(g(x))\neq y)$; by definition of $P(f,g)$: \lref{assumep} \label{pequiv}\nl

        Let $w\in D$ be such that $\forall x\in D. (f(g(x))\neq w)$; instantiation \lref{pequiv}\label{lety}\nl

        $(\forall u\in D. f(u)\neq w) \oor \nnot(\forall u\in D. f(u)\neq w)$; tautology \label{casetautology}\nl

        Case 1: Assume $\forall u\in D. f(u)\neq w$; \label{case1}\nl

        \n $\forall u\in D. (f(u)\neq w)\oor \forall u\in D. (g(u)\neq w)$; proof of disjunction: \lref{case1}\label{case1a} \nl

        $\exists w\in D. [\forall u\in D. (f(u)\neq w)\oor \forall u\in D. (g(u)\neq w)]$; construction: \lref{lety}, \lref{case1a}\label{case1b1}\nl

        $\exists v\in D. [\forall u\in D. (f(u)\neq v)\oor \forall u\in D. (g(u)\neq v)]$; \\
        \qquad\qquad\qquad\qquad\qquad\qquad\qquad\qquad\qquad\qquad\qquad\qquad\quad substitution of bound variable $w$ for $v$: \lref{case1b1}\label{case1b}\nl

        \p $(\forall u\in D. f(u)\neq w)\iimplies\exists v\in D. [\forall u\in D. (f(u)\neq v)\oor \forall u\in D. (g(u)\neq v)]$; \\
        \quad\qquad\qquad\qquad\qquad\qquad\qquad\qquad\qquad\qquad\qquad\qquad\qquad\qquad\qquad\qquad\qquad\qquad direct proof: \lref{case1}, \lref{case1b} \label{tuatoimpy1}\nl

        Case 2: Assume $\nnot(\forall u\in D. f(u)\neq w)$;\label{case2}\nl

        \n $\nnot(\forall u\in D. f(u)\neq w)\iif (\exists u\in D. \nnot(f(u)\neq w))$; \\
        \qquad\qquad\qquad\qquad\qquad\qquad\qquad\qquad\qquad\qquad\quad\qquad\qquad\qquad\qquad tautology, negation of quantifier \label{case2a1}\nl

        $\exists u\in D. \nnot(f(u)\neq w)$; modus ponens: \lref{case2}, \lref{case2a1}\label{case2a11}\nl

        $[\exists u\in D. \nnot(f(u)\neq w)]\iif[\exists u\in D. \nnot(\nnot(f(u)=w))]$; \\
        \qquad\qquad\qquad\qquad\qquad\qquad\qquad\qquad\qquad\qquad\qquad\qquad\qquad\qquad\qquad\qquad tautology, definition of $\neq$\label{case2a2}\nl

        $\exists u\in D. \nnot(\nnot(f(u)=w))$; modus ponens: \lref{case2a11}, \lref{case2a2} \label{case2a22}\nl

        $[\exists u\in D. \nnot(\nnot(f(u)=w))]\iif[\exists u\in D. f(u)=w]$; \\
        \qquad\qquad\qquad\qquad\qquad\qquad\qquad\qquad\qquad\qquad\qquad\quad\qquad\qquad\qquad tautology, definition of negation \label{case2a3}\nl
        
        $\exists u\in D. f(u)=w$; modus ponens: \lref{case2a22}, \lref{case2a3} \label{case2b}\nl

        Let $z\in D$ be such that $f(z)=w$; instantiation: \lref{case2b} \label{letx} \nl

        \n To obtain a contradiction, assume $\nnot(\forall p\in D. z\neq g(p))$; \label{contra}\nl

        \n $\nnot(\forall p\in D. z\neq g(p))\iif (\exists p\in D. \nnot(z\neq g(p)))$; \\
        \qquad\qquad\quad\qquad\qquad\qquad\qquad\qquad\qquad\qquad\qquad\qquad\qquad\qquad\qquad tautology, negation of quantifier \label{contratauto}\nl

        $\exists p\in D. \nnot(z\neq g(p))$; modus ponens: \lref{contra}, \lref{contratauto} \label{contraexists1}\nl

        $(\exists p\in D. \nnot(z\neq g(p)))\iif (\exists p\in D. \nnot(\nnot(z=g(p))))$; \\
        \qquad\qquad\qquad\qquad\qquad\qquad\qquad\qquad\qquad\qquad\qquad\qquad\qquad\qquad\qquad\qquad tautology, definition of $\neq$ \label{contratauto2}\nl

        $\exists p\in D. \nnot(\nnot(z=g(p)))$; modus ponens: \lref{contraexists1}, \lref{contratauto2} \label{contratauto3}\nl

        $(\exists p\in D. \nnot(\nnot(z=g(p))))\iif(\exists p\in D. z=g(p))$; \\
        \qquad\qquad\quad\qquad\qquad\qquad\qquad\qquad\qquad\qquad\qquad\qquad\qquad\qquad\qquad tautology, definition of negation \label{contratauto4}\nl

        $\exists p\in D. z=g(p)$; modus ponens: \lref{contratauto3}, \lref{contratauto4} \label{contraexists}\nl

        Let $e\in D$ be such that $z=g(e)$; instantiation: \lref{contraexists} \label{instantiatee}\nl

        $f(g(e))\neq w$; specialization: \lref{instantiatee}, \lref{lety}\label{specializeytoe}\nl

        $f(z)\neq w$; substitution of an occurence of $g(e)$ by $z$: \lref{specializeytoe}, \lref{instantiatee} \label{contrax}\nl

        \p This is a contradiction: \lref{letx}, \lref{contrax}\label{contraoccur}\nl

        \p $\forall p\in D. z\neq g(p)$; proof by contradiction: \lref{contra}, \lref{contraoccur}\label{bound}\nl

        $\forall u\in D. z\neq g(u)$; substitution of bound variable $p$ for $u$: \lref{bound} \label{contradone}\nl

        $\forall u\in D. f(u)\neq z\oor \forall u\in D. g(u)\neq z$; proof of disjunction: \lref{contradone} \label{contradisjunc}\nl

        $\exists z\in D. [\forall u\in D. f(u)\neq z\oor \forall u\in D. g(u)\neq z]$; construction: \lref{letx}, \lref{contradisjunc}\label{hehe1}\nl

        $\exists v\in D. [\forall u\in D. f(u)\neq v\oor \forall u\in D. g(u)\neq v]$; \\
        \qquad\qquad\,\,\qquad\qquad\qquad\qquad\qquad\qquad\qquad\qquad\qquad\qquad substitution of bound variable $z$ for $v$: \lref{hehe1} \label{hehe}\nl

        \p $\nnot(\forall u\in D. f(u)\neq w)\iimplies \exists v\in D. [\forall u\in D. f(u)\neq v\oor \forall u\in D. g(u)\neq v]$; \\
        \qquad\qquad\qquad\qquad\qquad\qquad\qquad\qquad\qquad\qquad\qquad\qquad\qquad\qquad\qquad\qquad\qquad direct proof: \lref{case2}, \lref{hehe}\label{tuatoimpy2}\nl

        $\exists v \in D.[  \forall u \in D. (f(u) \neq v) \oor \forall u \in D. (g(u) \neq v)]$; proof by cases: \lref{casetautology}, \lref{tuatoimpy1}, \lref{tuatoimpy2}\label{lasttautomo}\nl

        $Q(f,g)$; by definition of $Q(f,g)$: \lref{lasttautomo}, \label{getq}\nl

        \p $P(f,g)\iimplies Q(f,g)$; direct proof: \lref{assumep}, \lref{getq}\label{almostinside}\nl

        \p $\forall g\in \mathcal{F}. [P(f,g)\iimplies Q(f,g)]$; generalization: \lref{letg}, \lref{almostinside}\label{inside}\nl
        
        \p $\forall f\in \mathcal{F}. \forall g\in \mathcal{F}. [P(f,g)\iimplies Q(f,g)]$; generalization: \lref{letf}, \lref{inside}\

    \end{formal}
\end{proof}

\newpage
2.
Recall that, if $p$ is a polynomial of degree $m \geq 1$, then there exist coefficients $a_i$ for  $0 \leq i \leq m$ such that $a_m \neq 0$ and, for all numbers $n$,
$$p(n) =  \sum_{i=0}^m a_i n^i.$$
Give a well-structured informal proof that, for any polynomial $p$ of degree at least 1 whose coefficients are natural numbers,
there is a natural number $n$ such that $p(n)$ is not prime.




\begin{proof}
    \begin{adjustwidth}{1cm}{}
    \end{adjustwidth}

    \begin{adjustwidth}{0.5cm}{}
        Assume $p$ is a polynomial of degree $m\geq 1$ whose coefficients are natural numbers;

        We consider 3 cases of $a_0\in\nats$: $a_0$ is 0, 1, or greater than 1;

        Case 1: $a_0 = 0$;
        \begin{adjustwidth}{1.25cm}{}
            Let $n = 0\in\nats$;

            $\displaystyle p(n)=\sum_{i=0}^{m}a_i n^i=a_0\cdot 0^0+\sum_{i=1}^{m}a_i 0^i=a_0;$

            Since $0$ is not prime, this implies $p(n)=a_0=0$ is also not prime;
        \end{adjustwidth}
        For Case 1, we have shown that there exists a natural number $n$ such that $p(n)$ is not prime;

        Case 2: $a_0 = 1$;
        \begin{adjustwidth}{1.25cm}{}
            Let $n = 0\in\nats$;

            $\displaystyle p(n)=\sum_{i=0}^{m}a_i n^i=a_0\cdot 0^0+\sum_{i=1}^{m}a_i 0^i=a_0;$

            Since $1$ is not prime, this implies $p(n)=a_0=1$ is also not prime;
        \end{adjustwidth}
        For Case 2, we have shown that there exists a natural number $n$ such that $p(n)$ is not prime;

        Case 3: $a_0 \geq 2$;
        \begin{adjustwidth}{1.25cm}{}
            Let $n = a_0\in \nats$;

            $\displaystyle
                p(n)=\sum_{i=0}^m a_i a_0^i =a_0\cdot a_0^0 + \sum_{i=1}^m a_i a_0^i=a_0 + a_0\sum_{i=1}^m a_i a_0^{i-1} = a_0\left(1+\sum_{i=1}^m a_i a_0^{i-1}\right);
            $

            Now, since the addition and multiplication of natural numbers also produce natural numbers, we have $p(n)\in\nats$ and $1+\displaystyle\sum_{i=1}^m a_i a_0^{i-1}\in\nats$;

            Moreover, because $a_0^0=1\geq1$ and $a_0^{p+1}\geq a_0^{p}$ for all $p\in\nats$, combining with the fact that the product of some nonnegative numbers is also nonnegative, we have $\left(1+\displaystyle\sum_{i=1}^m a_i a_0^{i-1}\right) \geq (1+ a_m a_0^{m-1})$;

            Since $a_m\neq0$ implies $a_m\geq1$, and $m-1\geq0$, we also have that $1+ a_m a_0^{m-1} \geq (1+a_m)\geq 2>1$;
            

            Hence, because both $a_0$ and $\left(1+\displaystyle\sum_{i=1}^m a_i a_0^{i-1}\right)$ are natural numbers greater than 1, this means $p(n)=p(a_0)$ can be compose into a product of two natural numbers greater than 1, which implies $p(n)$ is not prime;
        \end{adjustwidth}
        For Case 3, we have shown that there exists a natural number $n$ such that $p(n)$ is not prime;

        For all cases, we have shown that there is such natural number $n$ which makes $p(n)$ not a prime;
    \end{adjustwidth}
    Therefore, we conclude if $p$ is a polynomial of degree $m\geq1$ whose coefficients are natural numbers, then there is a natural number $n$ such that $p(n)$ is not prime.
\end{proof}

\end{document}