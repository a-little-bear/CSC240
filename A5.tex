\documentclass[11pt]{alittlebear}


\def\name{Joseph Siu}
\def\course{CSC240 Winter 2024}
\def\headername{Homework Assignment }
\def\headernum{5}

\usepackage{comment}

\begin{document}
\newn{
    My name and student number: Joseph Siu, 1010085701.

    Sanchit Manchanda, Sepehr Jafari.
}
\begin{comment}
\newq{q1}{
    Give a well-structured informal proof by {\bf strong induction} that for any propositional formula $f$ built from propositional variables using only conjunction, disjunction, and negation, there exists a propositional formula $g$ in negation normal form that is logically equivalent to $f$ and has the same number of occurrences of propositional variables as in $f$.
}
\end{comment}
\newm{
    Recall the recursively defined set $NNF$ of propositional formulas in negation normal form as follows:

    Base Case: For any propositional variable $P$, $P \in NNF$, $\nnot(P)\in NNF$.

    Constructor Case: If $f,f'\in NNF,$ then $(f\aand f')\in NNF$ and $(f\oor f')\in NNF$.\\

    Consider the recursively defined set $S$ as follows:

    Base Case: For any propositional variable $P$, $P \in S$.

    Constructor Case: If $f, f'\in S$, then $(f \aand f')\in S$, $(f \oor f')\in S$, and $\nnot (f)\in S$.\\
}
\newl{l1}{
    If $g'\in NNF$, then there exist $g\in NNF$ such that $\nnot(g')\iif g$ and $g',g$ have the same number of variables.
}
\newp{[Proof of Lemma 1 by strong induction]\hfill

    For any propositional formula $f$, let $N_v(f)$=``\# of occurrences of propositional variables in $f$;'' let $N_c(f)$=``\# of connectors ($\aand,\oor,\nnot$) in $f$.''

    For any $n\in\Z^+$, let $M(n)=``\forall f\in NNF.O(n,f)''$ where $$O(n,f) = ``[n=N_v(f)\iimplies[\exists g\in NNF.(N_v(f)=N_v(g))\aand(\nnot(f)\iif g)]].''$$
    We now prove $\forall n\in\Z^+. M(n)$ by strong induction on $n$.
    \indenv{
        Base Case: $n=1$;

        \indenv{
            Let $f\in NNF$ be arbitrary, then either $1= N_v(f)$ or $1\neq N_v(f)$;

            Case 1: $1\neq N_v(f)$;
            \indenv{
                Since $1\neq N_v(f)$, the implication of $O(1,f)$ is vacuously true. 
            }
            For Case 1 we have shown $O(1,f)$.

            Case 2: $1=N_v(f)$;
            \indenv{
                There can be only 2 possible cases for $f$: $f=P$ and $f=\nnot(P)$ for some propositional variable $P$.

                Case 2.1: $f=P$;
                \indenv{
                    Let $g=\nnot(P)\in NNF$;

                    Since $N_v(f)=N_v(g)$, and $\nnot(f)=\nnot(P)\iif \nnot(P)=g$, we have $O(1,f)$.
                }
                For Case 2.1 we have shown $O(1,f)$.

                Case 2.2: $f=\nnot(P)$;
                \indenv{
                    Let $g=P\in NNF$;

                    Since $N_v(f)=N_v(g)$, and $\nnot(f)=\nnot(\nnot(P))\iif P=g$, we have $O(1,f)$.
                }
                For Case 2.2 we have shown $O(1,f)$.

                For all cases of Case 2 we have shown $O(1,f)$, thus $O(1,f)$.
                
            }
            For Case 2 we have shown $O(1,f)$.

            For all cases of $n=1$ we have shown $O(1,f)$, thus $O(1,f)$.
        }
        Since $f$ is arbitrary, we have shown $\forall f\in NNF. O(1,f)$, thus $M(1)$.

        \indenv{
            Let $n\in\Z^+-\{1\}$ be arbitrary;

            \indenv{
                Assume $\forall i\in\Z^+.[i<n\iimplies M(i)]$;

                \indenv{
                    Let $f\in NNF$ be arbitrary, then either $n= N_v(f)$ or $n\neq N_v(f)$;

                    Case 1: $n\neq N_v(f)$;
                    \indenv{
                        Since $n\neq N_v(f)$, the implication of $O(n,f)$ is vacuously true, thus $O(n,f)$.
                    }

                    For Case 1 we have shown $O(n,f)$.

                    Case 2: $n=N_v(f)$;
                    \indenv{
                        Since $N_v(f)=n>1$, this implies $f$ is not a propositional variable (not constructed from the base case), thus consider 2 cases (of the constructor cases): $f=(f'\aand f'')$ and $f=(f'\oor f'')$ for some $f',f''\in NNF$.

                        Case 2.1: $f=(f'\aand f'')$;
                        \indenv{
                            Since $1\leq N_v(f')<n$ and $1\leq N_v(f'')<n$, apply inductive hypothesis we have there exist $g',g''\in NNF$ such that $N_v(f')=N_v(g')$ and $N_v(f'')=N_v(g'')$ and $\nnot(f')\iif g'$ and $\nnot(f'')\iif g''$;

                            By Demorgan's Law, we have $\nnot(f)=\nnot(f'\aand f'')\iif (\nnot(f')\oor \nnot(f''))$;

                            By substition of equivalent formula, we have $\nnot(f)\iif (g'\oor g'')$ where $g'\oor g''\in NNF$ by our constructor cases;

                            Let $g=(g'\oor g'')\in NNF$, then $N_v(g)=N_v(g'\oor g'')=N_v(g')+N_v(g'')=N_v(f') + N_v(f'')=N_v(f)$;
                        }
                        Thus, for Case 2.1 we have $O(n,f)$.

                        Case 2.2: $f=(f'\oor f'')$;
                        \indenv{
                            Since $1\leq N_v(f')<n$ and $1\leq N_v(f'')<n$, apply inductive hypothesis we have there exist $g',g''\in NNF$ such that $N_v(f')=N_v(g')$ and $N_v(f'')=N_v(g'')$ and $\nnot(f')\iif g'$ and $\nnot(f'')\iif g''$;

                            By Demorgan's Law, we have $\nnot(f)=\nnot(f'\oor f'')\iif (\nnot(f')\aand \nnot(f''))$;

                            By substition of equivalent formula, we have $\nnot(f)\iif (g'\aand g'')$ where $g'\aand g''\in NNF$ by our constructor cases;

                            Let $g=(g'\aand g'')\in NNF$, then $N_v(g)=N_v(g'\aand g'')=N_v(g')+N_v(g'')=N_v(f') + N_v(f'')=N_v(f)$;
                        }
                        Thus, for Case 2.2 we have $O(n,f)$.

                        For all cases of Case 2 we have shown $O(n,f)$, thus $O(n,f)$.
                    }
                    For Case 2 we have shown $O(n,f)$.

                    For all cases of $N_v(f)$ we have shown $O(n,f)$, thus $O(n,f)$.
                }
                Since $f$ is arbitrary, we have shown $\forall f\in NNF. O(n,f)$, thus $M(n)$.
            }
        }
        By strong induction, we have shown $\forall n\in\Z^+. M(n)$.
        
    }
}
\newp{[Proof of Question 1 by strong induction]\hfill

    Consider $f\in S$, $n\in\N$, define $$Q(n, f)=``\exists g\in NNF. [N_c(f)=n\iimplies ((f \iif g) \aand (N_v(f)=N_v(g)))].''$$

    Consider $n\in\N$, define $P(n)$ = ``$\forall f\in S. Q(n, f)$.''

    To prove $\forall n\in\N. P(n),$ consider the base case when $n=0$. \
    \indenv{
        Let $f\in S$ be arbitrary;

        Consider 2 cases of $f$: $N_c(f)=0$ and $N_c(f)\neq0$.

        Case 1: Assume $N_c(f)=0$;
        \indenv{
            Since there is no connector in $f$, this implies $f$ is a propositional variable, which directly gives $f\in NNF$, thus $Q(0, f)$.
        }
        For Case 1 we have shown $Q(0, f)$.

        Case 2: Assume $N_c(f)\neq0$;
        \indenv{
            Since $N_c(f)\neq0$, the implication of $Q(0, f)$ is vacuously true. Moreover, because $NNF$ is non-empty, by picking any $g\in NNF$ we can conclude $Q(0, f)$.
        }
        For Case 2 we have shown $Q(0, f)$.

        Thus, since all cases are true, we have shown $Q(0,f)$.
    }
    Since $f$ is arbitrary, we have shown $\forall f\in S. Q(0, f)$, thus $P(0)$.
    \indenv{
        Now, let $n\in\Z^+$ be arbitrary;
        \indenv{
            Assume $\forall i\in\N. [i<n\iimplies P(i)]$;

            \indenv{
                Let $f\in S$ be arbitrary;

                Consider 2 cases of $f$: $N_c(f)=n$ and $N_c(f)\neq n$.

                Case 1: Assume $N_c(f)=n$;
                \indenv{
                    Consider 2 cases of $f$: $f$ is a propositional variable and $f$ is not a propositional variable (i.e. a formula with connectors).

                    Case 1.1: Assume $f$ is a propositional variable;
                    \indenv{
                        Since $f$ is a propositional variable, $f\in NNF$, thus $Q(n, f)$.
                    }
                    For Case 1.1 we have shown $Q(n, f)$.

                    Case 1.2: Assume $f$ is not a propositional variable;
                    \indenv{
                        Since $f\in S$ and is not constructed from the base case, thus consider 2 cases: $[\exists f'\in S. f=\nnot(f')]$, and $[\exists f'\in S.\exists f''\in S.f=(f'\star f'')]$ where $\star\in\{\aand,\oor\}$.

                        Case 1.2.1: Assume $\exists f'\in S. f=\nnot(f')$;
                        \indenv{
                            This implies $N_c(f')=n-1\geq0$, thus by specialization of our inductive hypothesis, $Q(n-1,f')$;

                            By instantiation, let $g'\in NNF$ be such that $N_c(f')=n-1\iimplies ((f' \iif g') \aand (N_v(f')=N_v(g')))$; 

                            Since $g'\in NNF$, by Lemma \ref{lemma:l1} and Demorgan's Law, we have $\exists g\in NNF. (\nnot(g')\iif g) \aand (N_v(g)=N_v(g'))$, let $g\in NNF$ be such formula;

                            By use of conjunction $\nnot(f')\iif g$; and by substitution of equivalent formula $f=\nnot(f')=\nnot(g')$, we have $f\iif g$;

                            By use of conjunction, we have $N_v(f)=N_v(f')=N_v(g')=N_v(g)$;
                        }
                        For Case 1.2.1 we have shown $Q(n, f)$.

                        Case 1.2.2: Assume $\exists f'\in S.\exists f''\in S.f=(f'\star f'')$;
                        \indenv{
                            Since $0\geq N_c(f')<n$ and $0\geq N_c(f'')<n$, by specialization of our inductive hypothesis, $Q(n-1,f')$ and $Q(n-1,f'')$;

                            By instantiation, let $g'\in NNF$ be such that $N_c(f')=n-1\iimplies ((f' \iif g') \aand (N_v(f')=N_v(g')))$;

                            By instantiation, let $g''\in NNF$ be such that $N_c(f'')=n-1\iimplies ((f'' \iif g'') \aand (N_v(f'')=N_v(g'')))$;

                            Since $g'\in NNF$ and $g''\in NNF$, by definition of $NNF$, $(g'\star g'')\in NNF$;

                            Now we have $f=(f'\star f''), (f'\star f'') \iif (g'\star g''), $ and $(g'\star g'') \in NNF$;

                            Also, $N_v(f)=N_v(f')+N_v(f'')=N_v(g')+N_v(g'')=N_v(g)$;
                        }
                        For Case 1.2.2 we have shown $Q(n, f)$.

                        For all cases of Case 1.2 we have shown $Q(n, f)$, thus $Q(n, f)$.
                    }
                    For Case 1.2 we have shown $Q(n, f)$.

                    For all cases of Case 1 we have shown $Q(n, f)$, thus $Q(n, f)$.
                }
                For Case 1 we have shown $Q(n, f)$.

                Case 2: Assume $N_c(f)\neq n$;
                \indenv{
                    Since $N_c(f)\neq n$, the implication of $Q(n, f)$ is vacuously true, moreover, since $NNF$ is non-empty, by picking any $g\in NNF$ we can conclude $Q(n, f)$.
                }
                For Case 2 we have shown $Q(n, f)$.

                For all cases of $f$ we have shown $Q(n, f)$, thus $Q(n, f)$.
            }
            Since $f$ is arbitrary, we have shown $\forall f\in S. Q(n, f)$, thus $P(n)$.
        }
    }
    By strong induction, we have shown $\forall n\in\N. P(n)$.
}
\newpage
\begin{comment}
\newn{
    Consider the following game between Alice and Bob.
    A \textit{position} in the game is defined by a triple $(X,Y,t)$, where 
    $X$ and $Y$ are disjoint sets of truth assignments to a set of $n$ propositional variables $\{P_i \mid 1 \leq i \leq n\}$ and $t  \in \ints^+$.
    The initial position of the game is decided in advance by a referee.

    The game proceed by rounds. Each round begins with a move by Alice followed by a response from Bob.
    Starting from the current position $(X,Y,t)$ of a round,
    Alice first chooses one of $X$ or $Y$ (say she chooses the set $X$).
    She then picks subsets $X'$ and $X''$ such that $X = X' \cup X''$ and integers $t' , t'' \in \ints^+$  such that $t = t' + t''$.
    She hands the triples $(X', Y, t')$ and $(X'', Y, t'')$ to Bob, who respond to this move by choosing either $(X', Y, t')$ or $(X'', Y, t'')$ as the new position for the next round.

    The game ends when it reaches a position $(X^*, Y^*, 1)$, where $X^* \subseteq X$ and $Y^* \subseteq Y$.
    Alice wins the game if there exists some $1 \leq i \leq n$ such that $\tau(P_i) \neq \tau'(P_i)$ for all $\tau \in X^*$ and $\tau' \in Y^*$.
    Otherwise, Bob wins the game.

    We say that a position $(X,Y,t)$ is \textit{promising} if Alice has a winning strategy starting from position $(X,Y,t)$. A winning strategy means that  Alice can choose her moves (which may depends on the responses from Bob) in a way such that she is guaranteed to win the game.
}
\end{comment}
\begin{comment}
\newq[Part A]{q2}{
    Give a recursive definition for the set of all promising positions.
}
\end{comment}
\newm{Solution to Question 2 - Part A.\hfill

    Fix $n\in\Z^+$. Consider the recursively defined set $PP$ of promising positions as follows:

    Let $X,Y$ be sets of truth assignments to a set of $n$ propositional variables $\{P_i \mid 1 \leq i \leq n\}$, and let $t\in\ints^+$.

    Base Cases: 
    \begin{enumerate}
        \item If $X=\emptyset$ or $Y=\emptyset$, then $(X,Y,s)\in PP$ for all $s\in\Z^+$.
        \item If there exists $1\leq i\leq n$ such that $\tau(P_i) \neq \tau'(P_i)$ for all $\tau \in X$ and $\tau' \in Y$, then $(X,Y,s)\in PP$ for all $s\in\Z^+$.
    \end{enumerate}

    Constructor Cases: \begin{enumerate}
        \item If $(X,Y,t)\in PP$, then $(X,Y,t+1)\in PP$.
        \item If $(X,Y,t)\in PP$ and $(X,Y',t')\in PP$, then $(X,Y\cup Y',t+t')\in PP$.
        \item If $(X,Y,t)\in PP$ and $(X',Y,t')\in PP$, then $(X\cup X',Y,t+t')\in PP$.
    \end{enumerate}

    Base Case 1 makes sense since if one of $X,Y$ is empty then Alice wins vacuously due to the quantifiers. 
    
    Base Case 2 directly comes from the definition of Alice's winning condition (these cases do overlap thus ambiguous). 

    Constructor Case 1 is valid since we can simply seperate $(X,Y,t+1)$ into $(X,Y,t)$ and $(\nil,\nil,1)$, then by assumption and base cases both are promising positions. 
    
    Constructor Cases 2, 3 directly come from the definition of the game. 

    If $X,Y$ are disjoint, $X, Y'$ are disjoints, so is $X, Y\cup Y'$; same for $X\cup X', Y$.

    Note that we can also define the Base Case 1 to be only $(X,Y,1)$ instead of all $s\in\Z^+$ s.t. $(X,Y,s)$, becasue of our Constructor Case 1, this is valid. However, for the sake of convinience for the proofs relating to $PP$, we define it as such.
}
\tcbcnt{question}{1}
\begin{comment}
\newq[Part B]{q3}{
    Give a well-structured informal proof by  {\bf structural induction} that for any propositional formula $f$ in negation normal form, for any set $X$ of truth assignments that make $f$ true, for any set $Y$ of truth assignments that make $f$ false, and for any integer $t \geq N_v(f)$, $(X,Y,t)$ is a promising position.
}
\end{comment}
\newp{[Proof of Question 2 - Part B by structural induction]\hfill

    Consider $f\in NNF$, define $Q(f)$ = ``for any set $X$ of truth assignments that make $f$ true, for any set $Y$ of truth assignments that make $f$ false, and for any integer $t \geq N_v(f)$, $(X,Y,t)$ is a promising position.'' 

    Now we prove $\forall f\in NNF. Q(f)$ by structural induction on $f$.

    \indenv{
        Let $f\in NNF$ be arbitrary;

        Base Case: 
                
        \indenv{
            Let $X$ be an arbitrary set of truth assignments that make $f$ true;
            \indenv{
                Let $Y$ be an arbitrary set of truth assignments that make $f$ false;

                Consider 2 cases:
                        
                Case 1: $f=P_1$ for some propositional variable $P_1$;
                \indenv{
                    If $X=\nil$ or $Y=\nil$, then $(X,Y,t)$ is covered by base case 1 of the recursive definition of $PP$;

                    If $X\neq\nil$ and $Y\neq\nil$, it can only be the case that $X=\{\tau\}$ and $Y=\{\tau'\}$ for some truth assignments $\tau,\tau'$ such that $\tau(P_1)=T$ and $\tau'(P_1)=F$;

                    So, by letting $i=1$, we have constructed such index $i$ that $\tau(P_i) \neq \tau'(P_i)$ for all $\tau \in X$ and $\tau' \in Y$, thus $(X,Y,1)\in PP$;

                    Since $N_v(f)=1$, by constructor case 1 of $PP$, we have $(X,Y,t)\in PP$ for all $t\geq N_v(f)$;
                }
                For Case 1 we have shown $(X,Y,t)\in PP$ for all $t\geq N_v(f)$.

                Case 2: $f=\nnot(P_1)$ for some propositional variable $P_1$;
                \indenv{
                    If $X=\nil$ or $Y=\nil$, then $(X,Y,t)$ is covered by base case 1 of the recursive definition of $PP$;

                    If $X\neq\nil$ and $Y\neq\nil$, it can only be the case that $X=\{\tau\}$ and $Y=\{\tau'\}$ for some truth assignments $\tau,\tau'$ such that $\tau(P_1)=F$ and $\tau'(P_1)=T$;

                    So, by letting $i=1$, we have constructed such index $i$ that $\tau(P_i) \neq \tau'(P_i)$ for all $\tau \in X$ and $\tau' \in Y$, thus $(X,Y,1)\in PP$;

                    Since $N_v(f)=1$, by constructor case 1 of $PP$, we have $(X,Y,t)\in PP$ for all $t\geq N_v(f)$;
                }
                For Case 2 we have shown $(X,Y,t)\in PP$ for all $t\geq N_v(f)$.
            }
            Since $Y$ is arbitrary, we have shown for all such $Y$, $ (X,Y,t)\in PP$ for all $t\geq N_v(f)$.
        }
        Since $X$ is arbitrary, we have shown for all such $X$, for all such $Y$, $(X,Y,t)\in PP$ for all $t\geq N_v(f)$. Thus, $Q(f)$.

        Constructor Cases: 
        \indenv{
            Assume $Q(f')$ and $Q(f'')$ for $f',f''\in NNF$ such that $f=(f'\star f''),$ where $\star\in\{\aand,\oor\}$.
            \indenv{
                Let $X$ be an arbitrary set of truth assignments that make $f$ true;
                \indenv{
                    Let $Y$ be an arbitrary set of truth assignments that make $f$ false;

                    Consider 2 cases of $\star$: $(\star=\aand)$ and $(\star=\oor)$.

                    Case 1: Assume $\star=\aand$;
                    \indenv{
                        By assumption, for all set $X'$ of truth assignments that make $f'$ true, for all set $Y'$ of truth assignments that make $f'$ false, and for all integer $t'\geq N_v(f')$, $(X',Y',t')$ is a promising position;

                        By assumption, for all set $X''$ of truth assignments that make $f''$ true, for all set $Y''$ of truth assignments that make $f''$ false, and for all integer $t''\geq N_v(f'')$, $(X'',Y'',t'')$ is a promising position;

                        For any $\tau\in X$, to make $f$ true, $\tau$ must make $f'$ true and $f''$ true, thus $\tau\in X'$ and $\tau\in X''$. This implies $X\subseteq X'\cap X''$;

                        For any $\tau'\in Y$, to make $f$ false, $\tau'$ must make $f'$ false or $f''$ false, thus $\tau'\in Y'$ or $\tau'\in Y''$. This implies $Y\subseteq Y'\cup Y''$;

                        By specialization, and $X\subseteq X'\cap X''$ and $Y\subseteq Y'\cup Y''$, we fix $X'=X''=X$, and choose $Y',Y''$ such that $Y=Y'\cup Y''$;

                        Then, by constructor case 2 of $PP$, we have $(X,Y',t')=(X',Y',t')\in PP$ and $(X,Y'',t'')=(X'',Y'',t'')\in PP$, which implies $(X,Y'\cup Y'',t'+t'')=(X,Y,t'+t'')\in PP$. Moreover, $t'+t''\geq N_v(f') + N_v(f'') = N_v(f)$;

                        By constructor case 1 of $PP$, we have $(X,Y,t)\in PP$ for all $t\geq N_v(f)$;
                    }
                    For Case 1 we have shown $(X,Y,t)\in PP$ for all $t\geq N_v(f)$.

                    Case 2: Assume $\star=\oor$;
                    \indenv{
                        By assumption, for all set $X'$ of truth assignments that make $f'$ true, for all set $Y'$ of truth assignments that make $f'$ false, and for all integer $t'\geq N_v(f')$, $(X',Y',t')$ is a promising position;

                        By assumption, for all set $X''$ of truth assignments that make $f''$ true, for all set $Y''$ of truth assignments that make $f''$ false, and for all integer $t''\geq N_v(f'')$, $(X'',Y'',t'')$ is a promising position;

                        For any $\tau\in X$, to make $f$ true, $\tau$ must make $f'$ true or $f''$ true, thus $\tau\in X'$ or $\tau\in X''$. This implies $X\subseteq X'\cup X''$;

                        For any $\tau'\in Y$, to make $f$ false, $\tau'$ must make $f'$ false and $f''$ false, thus $\tau'\in Y'$ and $\tau'\in Y''$. This implies $Y\subseteq Y'\cap Y''$;

                        By specialization, and $X\subseteq X'\cup X''$ and $Y\subseteq Y'\cap Y''$, we fix $Y'=Y''=Y$, and choose $X',X''$ such that $X=X'\cup X''$;

                        Then, by constructor case 3 of $PP$, we have $(X',Y,t')=(X',Y',t')\in PP$ and $(X'',Y,t'')=(X'',Y'',t'')\in PP$, which implies $(X'\cup X'',Y,t'+t'')=(X,Y,t'+t'')\in PP$. Moreover, $t'+t''\geq N_v(f') + N_v(f'') = N_v(f)$;

                        By constructor case 1 of $PP$, we have $(X,Y,t)\in PP$ for all $t\geq N_v(f)$;
                    }
                    For Case 2 we have shown $(X,Y,t)\in PP$ for all $t\geq N_v(f)$.

                    For all cases of $\star$ we have shown $(X,Y,t)\in PP$ for all $t\geq N_v(f)$.
                }
                Since $Y$ is arbitrary, we have shown for all such $Y$, $(X,Y,t)\in PP$ for all $t\geq N_v(f)$.

            }
            Since $X$ is arbitrary, we have shown for all such $X$, for all such $Y$, $(X,Y,t)\in PP$ for all $t\geq N_v(f)$. Thus, $Q(f)$.
        }
    }
    By structural induction, we have shown $\forall f\in NNF. Q(f)$.
}

\end{document}
