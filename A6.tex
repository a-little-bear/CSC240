\documentclass[11pt, brown, sepia, 1in]{hw}
\def\course{CSC240 Winter 2024}
\def\headername{Homework Assignment 6}
\def\name{Joseph Siu}

\usepackage{clsfiles/csc240}

\usepackage{comment}



\begin{document}
% \coverpage[clsfiles/stars]
\newn{
    My name and student number: Joseph Siu, 1010085701.

    Sanchit Manchanda, Sepehr Jafari.
}

\newq{1}{
    For $n \in \ints^+$, let $[n]$ denote the set $\{i  \in \ints^+ \ |\ i \leq n\}$.\\
    For each $n \in \ints^+$, each function $f:[n]\rightarrow \{0,1\}$, and each non-empty subset $I \subseteq [n]$, define the {\em restriction} of $f$ to $I$ to be the function $\restr{f}{I}:I \rightarrow \{0,1\}$ where, for each $x \in I$,
    $$
    \restr{f}{I}(x) = f(x).
    $$

    Give a well-structured informal proof using double induction that, for each $k \in \ints^+$, each $n \in \ints^+$, and each subset $F$ of functions from $[n]$ to $\{0,1\}$,
    if $n \geq k$ and
    $$
    |F| > \sum_{i=0}^{k-1} \binom{n}{i},
    $$
    then there exists a subset $I \subseteq [n]$ with $|I| = k$ such that $\{\restr{f}{I} : f \in F\}$ is the set of all functions from $I$ to $\{0,1\}$.

    You may use the following fact, known as Pascal's Identity, without proof.

    {\tbf{ Lemma}}:
    $\forall k \in \ints^+. \forall n \in \ints^+. \left[\binom{n}{k} = \binom{n-1}{k} + \binom{n-1}{k-1}\right].$
}
\newl{1}{
    For all $n\in\Z^+$, for all $k\in\Z^+$, if $n\geq k$, then
    \[\sum_{i=0}^{k-1} \binom{n}{i}\geq 2^k - 1.\]
}

\newp{[Proof of Lemma \ref{lemma:l1} by double induction]\hfill

    For all $n\in \Z^+$, for all $k\in \Z^+$, define the predicates $P(n)$, and $Q(n,k)$ as follows:

    $Q(n,k) = ``n\geq k\iimplies \sum_{i=0}^{k-1} \binom{n}{i}\geq 2^k-1$'';

    $P(n) = ``\forall k\in \Z^+. Q(n,k)''$.

    We will prove $\forall n\in \Z^+. P(n)$ by double induction.

    Base Case: Consider $n=1$. 

    \indenv{
        Let $k\in\Z^+$ be arbitrary;

        \begin{proofcases}
            \case $1=k$.

            \indenv{
                In this case we have that $\sum_{i=0}^{1-1}\binom{1}{i}=\binom{1}{0}=1\geq 2^1-1=1$. Thus, $Q(1,k)$ holds.
            }

            For Case 1, we have shown $Q(1,k)$.

            \case $1<k$.

            \indenv{
                In this case since $n<k$, the implication of $Q(1,k)$ is vacuously true. Thus, $Q(1,k)$ holds.
            }

            For Case 2, we have shown $Q(1,k)$.
        \end{proofcases}

        For all cases, we have shown $Q(1,k)$.
    }

    Since $k\in\Z^+$ was arbitrary, we have shown $P(1)$.

    \indenv{
        Let $n\in\Z^+$ be arbitrary;

        \indenv{
            Assume $P(n)$;

            Base Case: Since $\sum_{i=0}^{1-1}\binom{n+1}{i}=\binom{n+1}{0}=1\geq 2^1-1=1$, we have $Q(n+1,1)$.

            \indenv{
                Let $k\in\Z^+$ be arbitrary;

                \indenv{
                    Assume $Q(n+1,k)$;

                    \begin{proofcases}
                        \case $n+1<k+1$.
                        \indenv{
                            In this case, since $n+1<k+1$, the implication of $Q(n+1,k+1)$ is vacuously true. Thus, $Q(n+1,k+1)$ holds.
                        }

                        For Case 1, we have shown $Q(n+1,k+1)$.

                        \case $n+1\geq k+1$.
                        \indenv{
                            First by Lemma we have $\sum_{i=0}^k\binom{n+1}{i}=\binom{n+1}{0}+\sum_{i=1}^k\binom{n+1}{i}=\binom{n+1}{0}+\sum_{i=1}^k\bra{\binom{n}{i}+\binom{n}{i-1}}=1+\sum_{i=1}^k\binom{n}{i}+\sum_{i=1}^{k}\binom{n}{i-1}=\binom{n}{0}+\sum_{i=1}^k\binom{n}{i}+\sum_{i=1}^{k}\binom{n}{i-1}=\sum_{i=0}^k\binom{n}{i}+\sum_{i=0}^{k-1}\binom{n}{i}$.

                            Since we assumed $P(n)$, by specialization we have $\sum_{i=0}^{k}\binom{n}{i}\geq 2^{k+1}-1$ and $\sum_{i=0}^{k-1}\binom{n}{i}\geq 2^k-1$. Thus, we have $\sum_{i=0}^k\binom{n+1}{i}=\sum_{i=0}^{k}\binom{n}{i}+\sum_{i=0}^{k-1}\binom{n}{i}\geq 2^{k+1}-1+2^k-1\geq 2^{k+1}-1$ since $2^k\geq1$ for all $k\in\N$.
                        }

                        For Case 2, we have shown $Q(n+1,k+1)$.
                    \end{proofcases}

                    For all cases, we have shown $Q(n+1,k+1)$.

                }
            }

            By induction, we have shown $\forall k\in\Z^+. Q(n+1,k)$. Thus, we have shown $P(n+1)$.
        }
    }

    Therefore, by induction $\forall n\in\Z^+. P(n)$ holds.
}
\begin{comment}
    \newl{2}{
        For any $n\in\Z^+$, for any $F$ a subset of functions from $[n]$ to $\{0,1\}$ with $|F|\geq 2$, we have 
        \[\exists I\subseteq[n].I\neq\nil\iimplies[\forall f\in F. \forall f'\in F-\{f\}. \bra{\exists z\in I. \bra{\restr{f}{I}(z)\neq\restr{f'}{I}(z)}}].\]
    }
    
\newp{[Proof of Lemma \ref{lemma:l2} by contradiction]\hfill

    For all $n\in\Z^+$. For all subset of functions $F$ from $[n]$ to $\{0,1\}$ with $|F|\geq2$, define the predicate $P(n,F)=``\exists I\subseteq[n].I\neq\nil\iimplies[\forall f\in F.\forall f'\in F-\{f\}.\bra{\exists z\in I. \bra{\restr{f}{I}(z)\neq\restr{f'}{I}(z)}}].''$ 

    \indenv{
        Let $n\in\Z^+$ be arbitrary;
        \indenv{
            Let $F\subseteq \{0,1\}^{[n]}$ such that $|F|\geq2$ be arbitrary;
            
            \indenv{
                To obtain a contradiction, assume $\nnot (P(n,F))$, that is, $$\forall I\subseteq[n].I\neq\nil\aand[\exists f\in F.\exists f'\in F-\{f\}.\bra{\forall z\in I. \bra{\restr{f}{I}(z)=\restr{f'}{I}(z)}}].$$

                \indenv{
                    Let $I=[n]\neq\nil$, by specialization of our assumption, we have $\exists f\in F.\exists f'\in F-\{f\}.\bra{\forall z\in [n]. \bra{\restr{f}{[n]}(z)=\restr{f'}{[n]}(z)}}$.

                    Let $g\in F$ and $g'\in F-\{g\}$ be such that $\forall z\in [n]. \bra{\restr{g}{[n]}(z)=\restr{g'}{[n]}(z)}$.

                    Since all elements of $F$ are unique, we have $\forall f\in F.\forall f'\in F-\{f\}.\bra{\exists y\in[n]. f(y)\neq f'(y)}$.

                    By specialization of the above line, we have $\exists y\in[n]. g(y)\neq g'(y)$, this is equivalent to $\nnot\bra{\forall y\in[n]. g(y)=g'(y)}$.      

                    Since $I=[n]$, we have $\restr{g}{I}=g$ and $\restr{g'}{I}=g'$, thus by substitution we have $\nnot\bra{\forall z\in I.\restr{g}{I}(z)=\restr{g'}{I}(z)}$.
                }

                Since both $\bra{\forall z\in I. \restr{g}{I}(z)=\restr{g'}{I}(z)}$ and $\nnot\bra{\forall z\in I. \restr{g}{I}(z)=\restr{g'}{I}(z)}$, we have reached a contradiction.
            }

            Thus, by contradiction, we have shown $P(n,F)$.

        }

        Since $F$ was arbitrary, we have shown for any $F$ a subset of functions from $[n]$ to $\{0,1\}$ with $|F|\geq2$, we have $P(n,F)$.

    }

    Since $n$ was arbitrary, we have shown for any $n\in\Z^+$, for any $F$ a subset of functions from $[n]$ to $\{0,1\}$ with $|F|\geq2$, we have $P(n,F)$. This completes our proof (of lemma 2).
}
\end{comment}

\newp{[Proof of Question 1 by double induction]\hfill

    For all $k\in\Z^+$, for all $n\in\Z^+$, for all $F$ as a subset of functions from $[n]$ to $\{0,1\}$, define the predicates $P(k)$, $Q(n,k)$, $R(n,k,F)$ as follows:

    $R(n,k,F)=$$$``\sqrbra{\bra{n\geq k\aand |F|>\ds\sum_{i=0}^{k-1}\binom{n}{i}}\iimplies\bra{\exists I\subseteq[n]. (|I|=k) \aand  \{\restr{f}{I}: f\in F\}=\{0,1\}^{I}}}.''$$

    $Q(n,k) = ``\forall F\subseteq\{0,1\}^{[n]}.R(n,k,F)$

    $P(k) = ``\forall n\in\Z^+. Q(n,k)''$.

    We will prove $\forall k\in\Z^+. P(k)$ by double induction.

    Base Case: 
    
    Consider $k=1, n=1$. 

    \indenv{
        Let $F\subseteq\{0,1\}^{[1]}$ be arbitrary;

        \begin{proofcases}
            \case $|F|\leq1$.

            \indenv{
                Since $|F|\leq1=\sum_{i=0}^{1-1}\binom{1}{i}$, the implication of $R(1,1,F)$ is vacuously true.
            }

            For Case 1, we have shown R(1,1,F).

            \case $|F|>1$.

            \indenv{
                Since $F\subseteq\{0,1\}^{[1]}$, $|F|\geq2$, and $\abs{\{0,1\}^{[1]}}=2$, it follows that $F=\{0,1\}^{[1]}$. Now, let $F=\{f_1,f_2\}$ for some functions $f_1,f_2:[1]\to\{0,1\}$. By picking $I=[1]$, we have $|I|=1$, and $\{\restr{f}{I}: f\in F\}=F=\{0,1\}^{[1]}=\{0,1\}^{I}$.
            }

            For Case 2, we have shown R(1,1,F).
        \end{proofcases}

        For all cases, we have shown $R(1,1,F)$, thus we conclude $R(1,1,F)$.
    }

    Since $F$ was arbitrary, we have shown $Q(1,1)$.

    \indenv{
        Let $n\in\Z^+$ be arbitrary;

        \indenv{
            Assume $Q(n,1)$;

            \indenv{
                Let $F\subseteq\{0,1\}^{[n+1]}$ be arbitrary;

                \begin{proofcases}

                    \case $|F|\leq\sum_{i=0}^{1-1}\binom{n+1}{i}$.

                    \indenv{
                        The implication of $R(n+1,1, F)$ is vacuously true.
                    }

                    For Case 1 we have shown $R(n+1,1,F)$.

                    \case $|F|>\sum_{i=0}^{1-1}\binom{n+1}{i}$.
                    
                    \indenv{
                        Since $|F|>\sum_{i=0}^{1-1}\binom{n+1}{i}=1$, we have $|F|\geq2$. This implies there exists two distinct functions, namely $\exists f\in F.\exists f'\in F-\{f\}.\bra{\exists z\in[n+1]. f(z)\neq f'(z)}$. Let $g\in F$, $g'\in F-\{g\}$, and $z\in[n+1]$ be such instances, then by constructing $I=\{z\}\subseteq[n+1]$, we have $|I|=1$ and $\{\restr{f}{I}: f\in F\}=\{0,1\}^{I}$.
                    }


                    For Case 2 we have that $R(n+1,1,F)$.

                    %\indenv{
                    %By Lemma \ref{lemma:l2}, let $I\subseteq[n+1]$ be such that $\forall f\in F. \forall f'\in F-\{f\}. \bra{\exists z\in I. \bra{\restr{f}{I}(z)\neq\restr{f'}{I}(z)}}$.


                        %By specialization of our assumption $Q(n,1)$, we have $\exists I\subseteq[n]. (|I|=1) \aand  \curbra{\restr{\bra{\restr{f}{[n]}}}{I}: f\in F}=\{0,1\}^{I}$, let $I'$ be such $I\subseteq[n]$;

                        %Since $|I|=1$ and $\curbra{\restr{\bra{\restr{f}{[n]}}}{I'}: f\in F}=\{\restr{f}{I'}: f\in F\}$, by substitution, we have $\exists I'\subseteq[n]\subseteq[n+1]. (|I'|=1) \aand  \{\restr{f}{I'}: f\in F\}=\{0,1\}^{I'}$.
                    %}
                \end{proofcases}

                For all cases, we have shown $R(n+1,1,F)$, thus we conclude $R(n+1,1,F)$.
            }

            Since $F$ was arbitrary, we have shown $Q(n+1,1)$.
        }
    
    }

    By induction, we have shown $\forall n\in\Z^+. Q(n,1)$. Thus, we have shown $P(1)$.

    \indenv{
        Let $k\in\Z^+$ be arbitrary;
        \indenv{
            Assume $P(k)$;

            Base Case: 
            \indenv{
                Let $F\subseteq\{0,1\}^{[1]}$ be arbitrary;

                Since $1<k+1$, this shows the implication of $R(1,k+1,F)$ is vacuously true.
            }

            Since $F$ was arbitrary, we have shown $Q(1,k+1)$.

            \indenv{
                Let $n\in\Z^+$ be arbitrary;
                \indenv{
                    Assume $Q(n,k+1)$;

                    \indenv{
                        Let $F\subseteq\{0,1\}^{[n+1]}$ be arbitrary;

                        \begin{proofcases}
                            \case $|F|\leq\sum_{i=0}^{k}\binom{n+1}{i}$.

                            \indenv{
                                The implication of $R(n+1,k+1,F)$ is vacuously true.
                            }

                            For Case 1 we have shown that $R(n+1,k+1,F)$.
                            
                            \case $|F|>\sum_{i=0}^{k}\binom{n+1}{i}$.

                            \indenv{
                                By Lemma this implies \begin{align*}
                                    |F| &> \sum_{i=0}^{k}\binom{n+1}{i}\\
                                    &=\binom{n+1}{0}+\sum_{i=1}^{k}\binom{n+1}{i}\\
                                    &=1+\sum_{i=1}^k\bra{\binom{n}{i}+\binom{n}{i-1}}\\
                                    &=\binom{n}{0}+\sum_{i=1}^k\binom{n}{i}+\sum_{i=1}^k\binom{n}{i-1}\\
                                    &=\sum_{i=0}^k\binom{n}{i}+\sum_{i=1}^{k}\binom{n}{i-1}\\
                                    &=\sum_{i=0}^k\binom{n}{i}+\sum_{i=0}^{k-1}\binom{n}{i}
                                \end{align*}

                                \subcase If $\abs{\curbra{\restr{f}{[n]}:f\in F}}>\ds\sum_{i=0}^k\binom{n}{i}$:

                                By Specialization of assumption $Q(n,k+1)$ this implies there exists a subset $I\subseteq[n]$ such that $|I|=k+1$ and $\{\restr{f}{I}: f\in F\}=\{0,1\}^{I}$ by assumption $Q(n,k+1)$.

                                For subcase 2.1 we have shown that $R(n+1,k+1,F)$.

                                \subcase If $\abs{\curbra{\restr{f}{[n]}:f\in F}}\leq\ds\sum_{i=0}^k\binom{n}{i}$:

                                By multipling both sides we have $-\abs{\curbra{\restr{f}{[n]}:f\in F}}\geq-\ds\sum_{i=0}^k\binom{n}{i}$

                                Add this inequality to the previous inequality at Case 2, we have $|F|-\abs{\curbra{\restr{f}{[n]}:f\in F}}>\sum_{i=0}^k\binom{n}{i}+\sum_{i=0}^{k-1}\binom{n}{i}-\sum_{i=0}^k\binom{n}{i}=\sum_{i=0}^{k-1}\binom{n}{i}$.

                                By Lemma 1 and above, since $|F|-\abs{\curbra{\restr{f}{[n]}:f\in F}}\geq\sum_{i=0}^{k-1}\binom{n}{i}+1\geq 2^k$, thus this means there are two disjoint subsets $I_1,I_2\subseteq F$ such that $|\{\restr{f}{[n]}: f\in I_1\}|=|\{\restr{f}{[n]}: f\in I_2\}|=2^k$ and $\{\restr{f}{[n]}: f\in I_1\}=\{\restr{f}{[n]}: f\in I_2\}$ so that the cardinality of $F$ can decrease at least $2^k$ when we are restricting the domains of the functions in $F$ to $[n]$.

                                Now, since $\{\restr{f}{[n]}:f\in I_1\}\subseteq \{\restr{f}{[n]}:f\in F\}$ and $|\{\restr{f}{[n]}: f\in I_1\}|=2^k$, this implies $|\{\restr{f}{[n]}:f\in F\}|\geq 2^k\geq\sum_{i=0}^{k-1}\binom{n}{i}+1$, hence by specialization of the assumption $P(k)$ we have $Q(n,k)$, hence $\exists I'\subseteq[n].|I'|=k\aand\{\restr{\bra{\restr{f}{[n]}}}{I'}: f\in F\}=\{0,1\}^{I'}$. 

                                By constructing $I=I'\cup\{n+1\}$, since $\{\restr{f}{I}: f\in F\}\subseteq\{0,1\}^{I}$ which implies $\abs{\{\restr{f}{I}: f\in F\}}\leq\abs{\{0,1\}^I}$ (there exists an injection), and $\abs{\{\restr{f}{I}: f\in F\}}\geq 2\cd2^{k}=2^{k+1}=\abs{\{0,1\}^I}$ by our disjoint $I_1,I_2$ (the cardinality of $\{\restr{f}{I}:f\in F\}$ is at least $\{0,1\}^I$). Since both sets are finite, so they must have the same number of elements. Moreover, because of $\{\restr{f}{I}: f\in F\}\subseteq\{0,1\}^{I}$, this also shows they are equal as sets. Hence, $|I|=k+1$ and $\{\restr{f}{I}: f\in F\}=\{0,1\}^{I}$.

                                Since we have constructed such $I$, for subcase 2.2 we have shown that $R(n+1,k+1,F)$.
                            }

                            For Case 2, we have shown that $R(n+1,k+1,F)$.
                        \end{proofcases}

                        For all cases We have shown that $R(n+1,k+1,F)$. Thus, we conclude $R(n+1,k+1,F)$.
                    }

                    Since $F$ was arbitrary, we have shown $Q(n+1,k+1)$.
                }
            }

            By induction, we have shown $\forall n\in\Z^+. Q(n,k+1)$. Thus, we have shown $P(k+1)$.
        }
    }

    By induction, we have shown $\forall k\in\Z^+. P(k)$.
}

\newq{2}{
    A {\em cyclic shift} of a sequence $\{s_i\}_{i=1}^n$ is a sequence $\{s'_i\}_{i=1}^n$ such that,
    for some $k \in [n]$ and
    %positive integer  $k \le n$,
    for all $1 \le i \le n$,  the $i$'th term of this sequence is $s'_i =  s_{((i + k -1) \bmod n)+ 1}$.\\
    For example, the sequence 3,4,5,1,2  is a cyclic shift of the sequence 1,2,3,4,5, where $k = 2$.

    The  {\em prefix sums} of a sequence $\{s_i\}_{i=1}^n$ of numbers
    are the numbers $\sum_{i=1}^m s_i$ for $1 \le m \le n$.
    For example, the prefix sums of the sequence 1,2,3,4,5 are  the numbers 1,3,6,10, and 15.

    For all $n \in \ints^+$, 
    let $\text{OE}_n$ denote the set of finite sequence $\{r_i\}_{i=1}^{2n}$ of integers such that
    \begin{itemize}
    \item
    $r_i > 0$ if $i$ is odd,
    \item
    $r_i < 0$ if $i$ is even, and
        \item $\sum\limits_{i=1}^{2n} r_i \geq 0$.
    \end{itemize}

    Using the well-ordering principle, give a well-structured informal proof that, for all $n \in \ints^+$ and all
    sequences $r \in \text{OE}_n$, there is a cyclic shift of $r$ all of whose prefix sums are non-negative.
}

\newl{2}{
    For any $n\in\Z^+$. For any $r=\{r_i\}_{i=1}^{2n}\in\text{OE}_n$. Let CS$(r)$ denote the set of all cyclic shift of $r$, then for any $r'=\{r_i'\}_{i=1}^{2n}\in\T{CS}(r)$, we have $\sum_{i=1}^{2n}r_i=\sum_{i=1}^{2n}r_i'$.
}

\newp{[Proof of Lemma 2]\hfill

    \indenv{
        Let $n\in\Z^+$ be arbitrary;
        \indenv{
            Let $r=\{r_i\}_{i=1}^{2n}\in\text{OE}_n$ be arbitrary;
            
            \indenv{
                Let $r'\in\T{CS}(r)$ be arbitrary;

                By definition of CS$(r)$, we have $\exists k\in[2n].\forall i\in[2n].r'_i=r_{((i+k-1)\bmod 2n)+1}$.
                
                We first show the function $f:[2n]\to[2n]$ defined by $f(i)=((i+k-1)\bmod 2n)+1$ is a bijective function.

                To this end, we show that $f$ is injective.
                \indenv{
                    Assume $f(i)=f(j)$ for some $i,j\in[2n]$.

                    Then, we have $((i+k-1)\bmod 2n)+1=((j+k-1)\bmod 2n)+1$;
                    
                    By cancellation, we have $(i+k-1)\bmod 2n=(j+k-1)\bmod 2n$;

                    This implies $i+k-1=j+k-1+2nm$ for some $m\in\Z$;

                    By cancellation, we have $i=j+2nm$;

                    \indenv{
                        To obtain a contradiction, assume $m\neq0$.

                        Then, we have $i=j+2nm\geq j+2n$ or $i=j+2nm\leq j-2n$.

                        \begin{proofcases}
                            \case If $i=j+2nm\geq j+2n$, then we have $i=j+2nm\geq j+2n$. This implies $i-j\geq2n$. However, since $i,j\in[2n]$, we notice $1\leq i\leq 2n$ and $1\leq j\leq 2n$, so $i-j\leq 2n-1<2n$. This is a contradiction.
                            \case If $i=j+2nm\leq j-2n$, then we have $i=j+2nm\leq j-2n$. This implies $i-j\leq-2n$. However, since $i,j\in[2n]$, we notice $1\leq i\leq 2n$ and $1\leq j\leq 2n$, so $i-j\geq 1-2n>-2n$. This is a contradiction.
                        \end{proofcases}

                        For all cases contradiction occured.
                    }

                    Hence, by contradiction we have shown $m=0$. This implies $i=j$.
                }
                Hence, by definition of injective, we have shown $f$ is injective.

                Now, we show that $f$ is surjective.
                \indenv{
                    Let $y\in[2n] $ be arbitrary.

                    \begin{proofcases}
                        \case If $y=1$, then we have $f(-k+1+2n)=(((-k+1+2n+k-1)\bmod 2n)+1)=((2n)\bmod 2n)+1=0+1=1$.

                        For Case 1 all $y\in[2n]$ can be achieved by $f$.

                        \case If $y\neq1$, that is, $y>1$.
                        \indenv{
                            \subcase If $y\geq k$, then we have $y-k\in[2n]$, and $f(y-k)=((y-k+k-1)\bmod 2n)+1=((y-1)\bmod 2n)+1=y-1+1=y$.

                            For this subcase $y$ can be achieved by $f$.

                            \subcase If $y<k$, then we have $1-2n\leq y-k < 0$ and so $1\leq y-k+2n< 2n$ and $y-k+2n\in[2n]$, and $f(y-k+2n)=((y-k+2n+k-1)\bmod 2n)+1=((y+2n-1)\bmod 2n)+1=y-1+1=y$

                            For this subcase $y$ can be achieved by $f$.
                        }

                        Since for all subcases of Case 2 $y$ can be achieved by $f$, this shows for Case 2 all $y\in[2n]$ can be achieved by $f$.
                    \end{proofcases}

                    For all cases, $y$ can be achieved by $f$.
                }

                Since $y$ was arbitrary, we have shown $f$ is surjective by definition.

                Since $f$ is both injective and surjective, we have shown $f$ is bijective.

                Hence $\sum_{i=1}^{2n}r_i=\sum_{i=1}^{2n}r_{f(i)}=\sum_{i=1}^{2n}r_i'$ since addition is commutative and $f$ is bijective.
            }

            Since $r'\in\T{CS}(r)$ was arbitrary, we have shown $\forall r'\in\T{CS}(r).\sum_{i=1}^{2n}r_i=\sum_{i=1}^{2n}r_i'$.
        }

        Since $r\in\text{OE}_n$ was arbitrary, we have shown $\forall r\in\text{OE}_n.\forall r'\in\T{CS}(r).\sum_{i=1}^{2n}r_i=\sum_{i=1}^{2n}r_i'$.
    }

    Since $n\in\Z^+$ was arbitrary, we have shown $\forall n\in\Z^+.\forall r\in\text{OE}_n.\forall r'\in\T{CS}(r).\sum_{i=1}^{2n}r_i=\sum_{i=1}^{2n}r_i'$.
}

\newp{[Proof of Question 2 by Well Ordering]\hfill

    For $n\in\Z^+$, let $P(n)$ denote the statement ``$\forall r\in\text{OE}_n.\exists r'\in\T{CS}(r).\forall m\in\Z^+.1\leq m\leq 2n.\sum_{i=1}^m r_i'\geq0$''. We will show $\forall n\in\Z^+.P(n)$ by well ordering.

    \indenv{
        To obtain a contradiction assume $\exists n\in\Z^+. \nnot(P(n))$, that is, there exists $n\in\Z^+$, and exists a sequence $r\in\text{OE}_n$ such that there is no cyclic shift of $r$ all of whose prefix sums are non-negative.

        Namely, for all sequences $r=\{r_i\}_{i=1}^{2n}$, let CS$(r)$ denote the set of all cyclic shift of $r$, then $\exists n\in\Z^+.\nnot(P(n))=$$$\exists n\in\Z^+.\exists r=\{r_i\}_{i=1}^{2n} \in \T{OE}_n. \forall \{r_i'\}_{i=1}^{2n}\in \T{CS}(r). \exists m\in\Z^+. 1\leq m\leq 2n \aand \sum_{i=1}^m r_i'<0.$$

        Let $C=\{2n\in\Z^+\mid \exists r=\{r_i\}_{i=1}^{2n} \in \T{OE}_n. \forall \{r_i'\}_{i=1}^{2n}\in \T{CS}(r). \exists m\in\Z^+. 1\leq m\leq 2n \aand \sum_{i=1}^m r_i'<0\}$. Then, $C\neq\nil$ by our assumptiom.

        Since $\N$ has a well ordering and $\Z^+\subseteq\N$, this implies $\Z^+$ has a well ordering, so there exists a smallest element $2n_0\in C$.
        
        \indenv{
            Let $2n_0\in C$ be such smallest element.

            \indenv{
                To obtain a contradiction, assume $n_0=1$;

                Then, for all sequences $r\in \T{OE}_n$, consider $r\in\T{CS}(r)$;

                By specialization in the condition of $C$, let $m\in\Z^+$ be such that $1\leq m\leq 2n_0=2 \aand \sum_{i=1}^m r_i<0$. Then, we show there is no such $m$ exists using 2 cases:

                \begin{proofcases}
                    \case $m=1$.
                    \indenv{
                        Since $r_1>0$ by definition of $r\in\T{OE}_n$, we have $\sum_{i=1}^1 r_i=r_1>0$, hence contradiction.
                    }
                    \case $m=2$.
                    \indenv{
                        Since $\sum_{i=1}^{2}r_i=\sum_{i=1}^{2n_0}r_i\geq0$ by definition of $r\in\T{OE}_n$, and we have $\sum_{i=1}^{m}r_i=\sum_{i=1}^{2}r_i<0$, hence contradiction.
                    }
                \end{proofcases}

                Since for all cases we have reached a contradiction, we conclude this is a contradiction.
            }

            Since $n_0=1$ is a contradiction, we have shown that $n_0\neq 1$. That is, $n_0\geq 2$.

            Let $r_0=\{r_i\}_{i=1}^{2n_0} \in \T{OE}_{n_0}$ be such that for all $\{r_i'\}_{i=1}^{2n_0}\in \T{CS}(r_0)$, there exists $m\in\Z^+$ such that $1\leq m\leq 2n_0$ and $\sum_{i=1}^m r_i'<0$.

            Let $\{r_i'\}_{i=1}^{2n_0}\in \T{CS}(r_0)$ be arbitrary such that $r_i'>0$ for $i\in[2n_0]$ and $i$ is odd. Then by specialization of the condition of $C$, we have $\exists m\in\Z^+. 1\leq m\leq 2n_0 \aand \sum_{i=1}^m r_i'<0$. 

            Define $C'=\{m\in\Z^+\mid 1\leq m\leq 2n_0 \aand \sum_{i=1}^m r_i'<0\}$. 
            
            Then, $C'\neq\nil$ by our assumption. 
            
            \indenv{
                Let $m_0\in C'$ be such that $m_0$ is the smallest element of $C'$. 

                \indenv{
                    To obtain a contradiction, assume $m_0$ is odd;

                    Then, by our assumption, we have $r_{m_0}>0$.

                    \begin{proofcases}
                        \case $m_0=1$.
                        \indenv{
                            Since $r_1>0$, we have $\sum_{i=1}^{m_0} r_i=r_1>0$.
                        }

                        This is contradicting the definition of $m_0$ having negative prefix sum.

                        \case $m_0>1$.
                        \indenv{
                            Since $\sum_{i=1}^{m_0}r_i'<0$ and $r_{m_0}>0$, we have $\sum_{i=1}^{m_0-1}r_i<0$. 
                        }

                        This is contradicting the definition of $m_0$ being the smallest such $m_0$.
                    \end{proofcases}

                    For all cases contradiction occured.
                }

                Thus, by contradiction, $m_0$ must be even.

                \indenv{
                    To obtain a contradiction, assume $m_0=2n_0$;

                    Then, we have $\sum_{i=1}^{2n_0} r_i'<0$.

                    However, by definition of $r_0\in\T{OE}_{n_0}$, we have $\sum_{i=1}^{2n_0} r_i\geq0$.

                    By Lemma 2, this is a contradiction.
                }

                We conclude $m_0\neq 2n_0$. That is, $m_0\leq 2n_0-1$.

                Since $m_0+1\leq 2n_0$, we have $2n_0-m_0\geq1$. Hence, define a sequence $\{s_i\}_{i=1}^{2n_0-m_0}$ such that $s_i=r_{i+m_0}$ for all $i\in[2n_0-m_0]$, here $2n_0-m_0$ is even since both $2n_0$ and $m_0$ are even;

                Then, since we assumed $n_0$ is the smallest element of $C$, this implies $2n_0-m_0\notin C$. That is, there exists a cyclic shift of $s$ all of whose prefix sums are non-negative. Namely, $\exists s'=\{s_i'\}_{i=1}^{2n_0-m_0}\in\T{CS}(s). \forall m\in\Z^+. 1\leq m\leq 2n_0-m_0\implies \sum_{i=1}^m s_i'\geq0$.

                Consider the sequence $t=\{t_i\}_{i=1}^{2n_0}=\{s_i'\}_{i=1}^{2n_0-m_0}\circ\{r_i'\}_{i=1}^{m_0}$. Since $t\in\T{CS}(r_0)$ by our construction of $s'$ and $s$, this implies there exists $m_1\in\Z^+. 1\leq m_1\leq 2n_0 \aand \sum_{i=1}^{m_1} t_i<0$.

                Since by definition of $r_0\in\T{OE}_{n_0}$ and Lemma 2, we have $\sum_{i=1}^{2n_0} t_i\geq0$. 
                
                Moreover, by our construction of $s'$ and $s$, we have $\sum_{i=1}^{p} t_i'\geq0$ for all $p\in[2n_0-m_0]$. Also, by definition of $m_0$, we have $\sum_{i=m_0+1}^{q} t_i\geq0$ for all $q\in[2n_0-1]-[m_0]$. Combining these 2 inequalities, we have $\sum_{i=1}^{p'} t_i\geq0$ for all $p'\in[2n_0-1]$. Combining with $\sum_{i=1}^{2n_0} t_i\geq0$ we have $\forall q'\in[2n_0].\sum_{i=1}^{q'} t_i\geq0$.
            }
        }

        This contradicts our definition of $C$ and constuction of $r_0$ where $\forall r'=\{r_i'\}_{i=1}^{2n_0}\in\T{CS}(r_0). \exists m\in\Z^+. 1\leq m\leq 2n_0 \aand \sum_{i=1}^m r_i'<0$.
    }

    Therefore, we conclude $\forall n\in\Z^+. P(n)$. That is, for all $n\in\Z^+$ and all sequences $r\in\text{OE}_n$, there is a cyclic shift of $r$ all of whose prefix sums are non-negative.
}




\end{document}