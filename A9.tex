\documentclass[11pt, sakura, night, 1in]{hw}
\def\course{CSC240 Winter 2024}
\def\headername{Homework Assignment 9}
\def\name{Joseph Siu}

\usepackage{clsfiles/csc240}
\usepackage[noend, nocap]{clsfiles/alg}
\usepackage{tikz}
\usetikzlibrary{automata,positioning}

\newop{\A}{A}

\begin{document}

My name and student number: Joseph Siu, 1010085701. Sanchit Manchanda, Ali Zaki Rashid.

1. Consider the following iterative algorithm that finds the length of the longest increasing subsequence in the array $\A[1..n].$

\newalg{
    \as \ag{L[1]}{1}
    \afor{\ag{i}{2}\;\TO\;$n$\;\DO}{
        \as \ag{L[i]}{1}
        \afor{\ag{j}{1}\;\TO\;$i-1$\;\DO}{
            \aif{($A[j] < A[i]$) and ($L[j] \geq L[i]$) then}{
                \ag{L[i]}{L[j] + 1}
            }
        }
    }
    \as \ag{m}{L[n]}
    \afor{\ag{i}{2}\;\TO\;$n-1$\;\DO}{
        \aif{$L[i] > m$ then}{
            \ag{m}{L[i]}
        }
    }
}

(a) Give a precise statement of what it means for this algorithm to be partially correct.

\Precon: $n$ is a positive integer, and $\A[1..n]$ is an array with elements from a totally ordered domain.

\Postcon: The array $\A[1..n]$ is unchanged, and $m$ is assigned with a positive integer which represents the length of the longest increasing subsequence in $\A[1..n]$.

Partially correct: If $n$ is a positive integer $n$, $\A[1..n]$ is an array with elements from a totally ordered domain, and the algorithm is executed and terminated, then $\A[1..n]$ is unchanged, and $m$ is assigned with a positive integer which represents the length of the longest increasing subsequence in $\A[1..n]$. 


(b) Prove that this algorithm is partially correct.

We will use L1, L2 to denote line 1, line 2, and so on.

We also assumed the line numbers of the pseudo-code are fixed (L6, L7, L8 instead of L7, L8, L9).

\tit{Proof of Question 1(b).}

\indenv{
    Assume $n$ is a positive integer, $\A[1..n]$ is an array with elements from a totally ordered domain, and the algorithm is executed and terminated. 
    
    Since there are no assignments to $\A[1..n]$ in the algorithm, $\A[1..n]$ is unchanged.

    For $l\in\N$. Let $Q(l)=``$immediately after the $l^{\T{th}}$ iteration, $L[l+1]$ contains the length of the longest increasing (finite) subsequence in $\A[1..(l+1)]$ that ends with the term $\A[l+1]''$; and let $P(l)$=``If the for-loop from line 2 to line 5 is executed at least $l$ times, then $Q(l)$.'' 

    \tbf{Lemma 1.} $\forall l\in\N. P(l).$

    \tit{Proof of Lemma 1 by strong induction.}

    \indenv{
        Let $l\in\N$ be arbitrary;
        \indenv{
            Assume $\forall k\in\N. (k<l)\iimplies P(k)$.
            \indenv{
                Assume the for-loop from line 2 to line 5 is executed at least $l$ times.

                \begin{proofcases}
                    \case $l=0$. 
                    \indenv{
                        Then trivially the length of the longest increasing subsequence in $\A[1..1]$ is 1, and we assigned $L[1]=1$ on L1. Thus $Q(0)$ holds.
                    }
                    For Case 1 $Q(l)$.

                    \case $l=1$. 
                    \indenv{
                        Then for the first iteration, $i=2$; on L3 $L[2]$ is assigned with 1; on L4 since $j$ is from 1 to $2-1=1$, we only execute L5 once where $j=1$, and there are 2 subcases due to $L[1]=1\ge1=L[2]$:
                    
                        \subcase $A[j]<A[i]$ 
                        \indenv{
                            Then on L5 $L[2]$ is assigned with $L[1]+1=2$, after this we end this iteration, and now $L[2]=2$ is indeed the length of the longest increasing subsequence in $\A[1..2]$. 
                        }

                        Thus $Q(l)$ holds for subcase 2.1.

                        \subcase $A[j]\ge A[i]$. 
                        \indenv{
                            Then no assignment on L5 has been made. After this we end this iteration, and now $L[2]=1$ is indeed the length of the longest increasing subsequence in $\A[1..2]$ since $\A[j]=\A[1]\ge\A[2]=\A[i]$. 
                        }

                        Thus $Q(l)$ holds for subcase 2.2.
                    }
                    
                    For all subcases of Case 2 we have shown $Q(l)$ holds, so for Case 2 $Q(l)$.

                    \case $l\ge2$. 
                    \indenv{
                        We first assign $L[l]=1$ on L3. 
                        
                        Now, let $S=\{p\in[l]\mid A[p]<A[l]\}$, and let $S'=\{L[p]\mid p\in S\}$. 
                        
                        \subcase $S=\emptyset$. 
                        \indenv{
                            Then since for all $p\in[l].\nnot(A[p]<A[l])$, no assignment on L5 has been made, and $L[l+1]=1$ is indeed the length of the longest increasing subsequence in $\A[1..(l+1)]$ with the last term $\A[l+1]$ (all previous terms are at least $\A[l+1]$). 
                        }
                        Thus $Q(l)$ holds for subcase 3.1.

                        \subcase $S\ne\emptyset$. 
                        \indenv{
                            Then by construction this implies $S'\ne\emptyset$.

                            $i=l+1$ on L2.
                            
                            Since $S'$ is a finite non-empty subset of $\Z^+$, we are allowed to construct $q=\max S'$. 
                            
                            Then $q\in S'$, by construction there exists $p'\in S$ such that $q=L[p']$, we instantiate such $p'$. 
                            
                            By inductive hypothesis (specialization and modus ponens), $q$ is the length of the longest increasing subsequence in $\A[1..p']$ that ends with the term $\A[p']$. We instantiate such subsequence as $\{s_o\}_{o=1}^q$. Moreover, because $q=\max S'$, this means $q$ is the length of the longest increasing subsequence in $\A[1..l]$ that ends with a term less than $\A[l]$.

                            Now, we claim the subsequence $\{s_o\}_{o=1}^q\circ\{\A[l+1]\}$ is the longest increasing subsequence in $\A[1..(l+1)]$ that ends with the term $\A[l+1]$. Indeed, firsly for a subsequence to be both increasing and ends with $\A[l+1]$, the term before $\A[l+1]$ must be less than $\A[l+1]$, and by A8 $\{s_o\}_{o=1}^q\circ\{\A[l+1]\}$ is increasing since $\{s_o\}_{o=1}^q$ is increasing and $s_q=\A[p']<\A[l+1]$ by definition of $S$. Secondly, obviously the concatenation shows the subsequence ends with $\A[l+1]$. Lastly, $\{s_o\}_{o=1}^q$ being the longest increasing subsequence in $\A[1..l]$ implies $\{s_o\}_{o=1}^q\circ\{\A[l+1]\}$ is the longest increasing subsequence in $\A[1..(l+1)]$ that ends with the term $\A[l+1]$. 

                            Therefore, as long as we show $L[l+1]=q+1$, we can conclude $Q(l)$ holds.

                            To this end, since $1\le p'\le l=i-1$, for the for-loop from L4 to L5, right after the iteration where $j=p'$, since $A[p']<A[l]$, and $L[p']\ge L[l]$ due to our construction of $q$, on L5 $L[l+1]$ is assigned with $L[p']+1=q+1$. And after this iteration, consider 2 cases:

                            1. If $A[j]<A[l]$, then since $L[x]\le L[p']=L[l+1]$ for all $x\in S'$ by construction of $q$, no assignment on L5 has been made.

                            2. If $A[j]\ge A[l]$, then no assignment on L5 has been made.

                            For both subcases 3.2.1 and 3.2.2, we have shown that immediately after the $l^{\T{th}}$ iteration, $L[l+1]=q+1$ is indeed the length of the longest increasing subsequence in $\A[1..(l+1)]$ that ends with the term $\A[l+1]$. Thus $Q(l)$ holds for subcase 3.2.
                        }
                    }

                    For all subcases of Case 3 we have shown $Q(l)$ holds, so for Case 3 $Q(l)$.
                \end{proofcases}

                For all cases we have shown $Q(l)$ holds, so $Q(l)$.
            }

            By direct proof, $P(l)$.
        }
    }

    By strong induction, $\forall l\in\N. P(l)$. \qed

    Now, since there are no assignments to $i$ and $j$ within their for-loop respectively, and so only finitely many for-loop iterations has performed thus the for loop from L2 to L5 eventually terminates. Now we are on L6.
    
    \begin{proofcases}
        \case $n=1$.
        \indenv{
            Then we assign $m$ with $L[n]=L[1]=1$ and the algorithm terminates. Thus $m$ is assigned with a positive integer which represents the length of the longest increasing subsequence in $\A[1..n]$.
        }
        For Case 1 $m$ is assigned with a positive integer which represents the length of the longest increasing subsequence in $\A[1..n]$.
        \case $n\ge2$.
        \indenv{
            From L6 to L8, we assign $m$ with the largest value in $L[2..n]$, since $L[n]\ge L[1]$, such $m$ is also the largest value in $L[1..n]$. Since by Lemma 1 $L[y]$ is the length of the longest increasing subsequence in $\A[1..y]$ ending with $\A[y]$ for all $y\in[n]$, so the maximum of $L[1..y]$ is indeed the length of the longest increasing subsequence in $\A[1..n]$. 
        }
        For Case 2 $m$ is assigned with a positive integer which represents the length of the longest increasing subsequence in $\A[1..n]$.
    \end{proofcases}

    For all cases $m$ is assigned with a positive integer which represents the length of the longest increasing subsequence in $\A[1..n]$, thus by proof of conjunction $\A[1..n]$ is unchanged, and $m$ is assigned with a positive integer which represents the length of the longest increasing subsequence in $\A[1..n]$.   
}

By direct proof, the algorithm is partially correct. \qed


2. For each $k \in \ints^+$, let 
$ X_k = \{x \in \{0,1\}^*  :  \nnot (\exists y \in \{0,1\}^k. (x = y\cdot y))\}$\\
and  consider the NFA $N_k = (Q, \{0,1\}, \de, q_{0}, F)$,   where:\\
$Q = \{ q_i :  0 \le i \le 2k+1 \} \cup \{p_i : 0 \le i \le k-1\} \cup \{z_i  : 0 \le i \le k-1\}$,\\
$F =  \{q_i  :  0 \le i \le 2k -1 \} \cup \{q_{2k+1}\}$,\\
$\de(q_i, 0) = \{q_{i+1}, z_0 \}$  for $0 \le i \le k-1$,\\
$\de(q_i, 1) = \{q_{i+1}, p_0 \}$  for $0 \le i \le k-1$,\\
$\de(q_i,0) = \de(q_i,1) = \{q_{i+1}\}$ for $k \le i \le 2k$,\\
$\de(q_{2k+1},0) = \de(q_{2k+1},1) = \{q_{2k+1}\}$, \\
$\de(z_i, 0) = \de(z_i,1) = \{z_{i+1}\}$  for  $0 \le i \le k-2$,\\
$\de(z_{k-1},1) =  \{q_{2k+1}\}$,\\
$\de(z_{k-1},0) =  \nil$,\\ 
$\de(p_i, 0) = \de(p_i,1) = \{p_{i+1}\}$  for  $0 \le i \le k-2$,\\
$\de(p_{k-1},0) =   \{q_{2k+1}\}$,\\
$\de(p_{k-1},1) =  \nil$, and\\
$\de(q,\lambda) = \nil$ for all $q \in Q$.\\

% Here is a drawing for $N_2$:


% \begin{tikzpicture}[shorten >=1pt,node distance=1.75cm,on grid,auto,initial text=]
%     \node[state,initial,accepting]   (q_0)                {$q_0$};
%     \node[state,accepting]           (q_1) [right=of q_0] {$q_1$};
%     \node[state,accepting]           (q_2) [right=of q_1] {$q_2$};
%     \node[state,accepting]           (q_3) [right=of q_2] {$q_3$};
%     \node[state]                     (q_4) [right=of q_3] {$q_4$};
%     \node[state,accepting]           (q_5) [right=of q_4] {$q_5$};
%     \node[state]                     (p_0) [above=of q_0] {$p_0$};
%     \node[state]                     (p_1) [right=of p_0] {$p_1$};
%     \node[state]                     (z_0) [below=of q_0] {$z_0$};
%     \node[state]                     (z_1) [right=of z_0] {$z_1$};
%     \path[->]
%      (q_0)   edge                node {0,1} (q_1)
%      (q_1)   edge                node {0,1} (q_2)
%      (q_2)   edge                node {0,1} (q_3)
%      (q_3)   edge                node {0,1} (q_4)
%      (q_0)   edge                node {1}   (p_0)
%      (q_1)   edge                node[above] {1}   (p_0)
%      (q_0)   edge                node[left] {0}   (z_0)
%      (q_1)   edge                node {0}   (z_0)
%      (q_4)   edge                node {0,1} (q_5)
%      (p_0)   edge                node {0,1} (p_1)
%      (z_0)   edge                node[below] {0,1} (z_1)
%      (p_1)   edge [bend left, above] node {0}   (q_5)
%      (z_1)   edge [bend right, below] node {1}   (q_5)
%      (q_5)   edge [loop above]   node {0,1} (q_5);
%  \end{tikzpicture}

(a) For each state $q \in Q$, describe the set of strings $w \in \{0,1\}^*$ such that $q \in  \de^*(q_0, w)$.\\
Your descriptions should not mention $\delta$.

\tbf{Description.} We will use ``letter'' to denote a single element in $\Si=\{0,1\}$.

Let $k\in\Z^+$ be arbitrary. By definition of $\de^*$, it is equivalent to describe for each $q\in Q$, what strings $w\in\{0,1\}^*$ will lead to state $q$ from the initial state $q_0$. We will prove our description in part (b).

Now, if $q=q_i$ for some $i\in[2k]\cup\{0\}$ (we instantiate such $i$), then $q$ can be reached from $q_0$ by a string $w\in\{0,1\}^*$ if and only if $w$ is a string with length $i$.

If $q=q_{2k+1}$, there are 2 cases that the string will reach $q_{2k+1}$: First, any string $w\in\{0,1\}^*$ with length at least $2k+1$ will reach $q_{2k+1}$; Second, if $w\in\{0,1\}^*$ has length at least $k+1$, and there exists 2 letters $a\in w$, $b\in w$ such that $a\ne b$ and they are $k-1$ letters apart, then $w$ will reach $q_{2k+1}$ (by ``$\in$'' we mean $a$ is one of the letters of the string $w$, etc).

Let $\om=\{w\in\{0,1\}^*\mid \T{ $w$ has length at most $k-1$}\}$.

For $p_0$, if $w=w'\cd 1$ for some $w'\in\om$, then $w$ will reach $p_0$; For $p_i$ with $i\in[k-1]$, if $w=w'\cd 1\cd w''$ for some $w'\in\om$ and $w''\in\{0,1\}^{i}$, then $w$ will reach $p_i$.

Similarly, for $z_0$, if $w=w'\cd 0$ for some $w'\in\om$, then $w$ will reach $z_0$; For $z_i$ with $i\in[k-1]$, if $w=w'\cd 0\cd w''$ for some $w'\in\om$ and $w''\in\{0,1\}^{i}$, then $w$ will reach $z_i$.

% First since all $q_i$ where $i\in[2k+1]\cup\{0\}-\{2k\}$ are final states, we can see all strings that have length not equal to $2k$ will be accepted. 

% Now we focus on strings $w\in\{0,1\}^*$ with length $2k$. Let $w\in\{0,1\}^{2k}$ be arbitrary, then we can think $w$ as a concatenation of $w=p\cd q$ where $p\in\{0,1\}^k$ and $q\in\{0,1\}^k$. We claim that string $w$ with length $2k$ is accepted

(b) Prove that $L(N_k) = X_k$ for all $k \in \ints^+$.

\tbf{Lemma 1.} For all $k\in\Z^+$. For all $x\in \{0,1\}^*-X_k$. $|x|=2k$.

\tit{Proof of Lemma 1.}
\indenv{
    Let $k\in\Z^+$ be arbitrary. 
    \indenv{
        Let $x\in \{0,1\}^*-X_k$ be arbitrary. 

        Since $X_k\subseteq\{0,1\}^*$, by definition of $X_k$ we have $\{0,1\}^*-X_k=\{x\in\{0,1\}^*: \exists y\in\{0,1\}^k. (x=y\cd y)\}$, since $x\in \{0,1\}^*-X_k$, we instantiate $y\in\{0,1\}^k$ such that $x=y\cd y$. 

        Since $|y|=k$ and $|y\cd y|=2k$, by substitution we have $|x|=2k$.
    }

    Since $x$ is arbitrary, we conclude for all $x\in \{0,1\}^*-X_k$. $|x|=2k$.
}

Since $k$ is arbitrary, we conclude Lemma 1 holds. \qed

\tit{Proof of Question 2(b).}
\indenv{
    Let $k\in\Z^+$ be arbitrary.
    \indenv{
        Let $w\in\{0,1\}^*$ be arbitrary.
        
        Consider 3 cases of the length of $w$: $|w| < 2k, |w|=2k, |w|>2k$.

        Since there are no $\la$ transitions, for all $q\in Q$ we have $\de(q,\la)=\nil$, so we will simply ignore checking $\la$ transitions.

        \begin{proofcases}
            \case $|w| < 2k$.
            \indenv{
                Since $\de(q_i,0)=\{q_{i+1},z_0\}$ and $\de(q_i,1)=\{q_{i+1},p_0\}$ for $0\le i \le k-1$, and $\de(q_i,0)=\de(q_i,1)=\{q_{i+1}\}$ for $k\le i\le 2k$, from these we have $q_{i+1}\in\de(q_i,\al)$ for $0\le i\le 2k$ where $\al\in\{0,1\}$.

                Hence, since $q_i$ is a final state for $i\in[2k+1]\cup\{0\}-\{2k\}$, the string $w$ starting from $q_0$ and through the walk $\{\om_j\}_{j\in[|w|]}, \om_j=(q_{j-1}, q_j)$ reaches a final state $q_{|w|}$ since $|w| < 2k$ and so $|w|\in[2k+1]\cup\{0\}-\{2k\}$ by our case assumption. 
            }

            For Case 1 we have shown that when $|w| < 2k$, $w$ is accepted by $N_k$, thus $w\in L(N_k)$. Moreover, by Lemma 1 and specialization of $k$, because $|w|\neq 2k$, thus $w\notin \{0,1\}^*-X_k$, so $w\in X_k$. Therefore $w\in L(N_k)$ if and only if $w\in X_k$.

            \case $|w| > 2k$.
            \indenv{
                $|w| > 2k$ implies $|w|\ge 2k+1$, so $|w| - 2k - 1 \ge 0$. We will use $[0]$ to denote the empty set.

                Since $\de(q_i,0)=\{q_{i+1},z_0\}$ and $\de(q_i,1)=\{q_{i+1},p_0\}$ for $0\le i \le k-1$, and $\de(q_i,0)=\de(q_i,1)=\{q_{i+1}\}$ for $k\le i\le 2k$, from these we have $q_{i+1}\in\de(q_i,\al)$ for $0\le i\le 2k$ where $\al\in\{0,1\}$.

                Moreover, since $\de(q_{2k+1},0)=\de(q_{2k+1},1)=\{q_{2k+1}\}$, and $q_{2k+1}$ is a final state, by constructing the walk $\{(q_{i-1},q_i)\}_{i\in[2k+1]}\circ\{(p_{j},p_{j})\}_{j\in[|w|-2k-1]}, p_j=q_{2k+1}$ for all $j\in[|w|-2k-1]$. We can see that $w$ is accepted by $N_k$ since it reaches the final state $q_{2k+1}$ through such walk.
            }

            For Case 2 we have shown that when $|w| > 2k$, $w$ is accepted by $N_k$, thus $w\in L(N_k)$. Moreover, by Lemma 1 and specialization of $k$, because $|w|\neq 2k$, thus $w\notin \{0,1\}^*-X_k$, so $w\in X_k$. Therefore $w\in L(N_k)$ if and only if $w\in X_k$.

            \case $|w|=2k$.
            \indenv{
                We will use the word ``character'' to denote element in $\Si=\{0,1\}$, moreover, since $w$ is a finite sequence of characters, we will use $w_i$ to denote the $i^{\T{th}}$ character of $w$ starting from position 1.

                \subcase $\nnot(\exists y\in\{0,1\}^k. (w=y\cd y))$.
                \indenv{
                    First, $w=y\cd y=yy$ is equivalent to $\forall i\in[2k]$. $w_i=yy_i$.

                    Hence, $\nnot(\exists y\in\{0,1\}^k. (w=y\cd y))$ is equivalent to $\forall y\in\{0,1\}^k.\exists i\in[2k]. w_i\ne yy_i$.

                    Let $y'\in\{0,1\}^k$ be such that $\forall i\in[k]. y'_i=w_i$. Then by specialization of the above formula, we have $\exists i\in[2k]. w_i\ne y'y'_i$.

                    Since by construction $\forall i\in[k]. w_i=y'y'_i$, combining with $\exists i\in[2k]. w_i\ne y'y'_i$ we have $\exists i\in[2k]-[k]. w_i\ne y'y'_i$, instantiate such $i$.

                    Since $w_i\ne y'y'_i=y'y'_{i-k}$ but by specialization $w_{i-k}=y'_{i-k}=y'y'_{i-k}$, by substitution we have $w_i\ne w_{i-k}$.

                    \tit{Subsubcase (a)} $w_i=0$.
                    \indenv{

                        Since $w_i\ne w_{i-k}$, we have $w_{i-k}=1$.

                        Now define $\forall m\in[2k-i], r_m=(q_{2k+1}, q_{2k+1})$, and construct the walk $\{(q_{j-1}, q_j)\}_{j\in [i-k-1]}\circ\{(q_{i-k-1}, p_0)\}\circ\{(p_{l-1}, p_l)\}_{l\in[k-1]}\circ\{(p_{k-1}, q_{2k+1})\}\circ\{r_m\}_{m\in[2k-i]}$. Here 
                        
                        $\{(q_{j-1}, q_j)\}_{j\in [i-k-1]}$ is valid since $q_{b+1}\in \de(q_b, 0)=\de(q_b, 1)$ for $0\le b\le 2k$ from Case 2; 
                        
                        $\{(q_{i-k-1}, p_0)\}$ is valid due to $\de(q_a,1)=\{q_{a+1}, p_0\}$ for $0\le a\le k-1$ and $w_{i-k}=1$; 
                        
                        $\{(p_{l-1}, p_l)\}$ is valid due to $\de(p_c,0)=\de(p_c,1)=\{p_{c+1}\}$ for $0\le c\le k-2$;
                        
                        $\{(p_{k-1}, q_{2k+1})\}$ is valid due to $\de(p_{k-1},0)=\{q_{2k+1}\}$ and $w_i=0$ (the length of the walk before is $(i-k-1)+(1)+(k-1)=i-1$, thus we read the character $w_i$ during $\{(p_{k-1}, q_{2k+1})\}$);
                        
                        and finally $\{r_m\}_{m\in[2k-i]}$ is valid due to $\de(q_{2k+1},0)=\de(q_{2k+1},1)=\{q_{2k+1}\}$.
                        
                        So, we have shown our walk is valid and reaches the final state $q_{2k+1}$. Note that all indexes are valid due to basic algebra and arithmetic manipulations.
                    }

                    For subsubcase (a) we have shown $w$ is accepted by $N_k$ thus $w\in L(N_k)$. And by definition of $X_k$, $w\in X_k$. So for subsubcase (a) $w\in L(N_k)$ if and only if $w\in X_k$.

                    \tit{Subsubcase (b)} $w_i=1$.
                    \indenv{
                        Since $w_i\ne w_{i-k}$, we have $w_{i-k}=0$.

                        Now define $\forall m\in[2k-i], r_m=(q_{2k+1}, q_{2k+1})$, and construct the walk $\{(q_{j-1}, q_j)\}_{j\in [i-k-1]}\circ\{(q_{i-k-1}, z_0)\}\circ\{(z_{l-1}, z_l)\}_{l\in[k-1]}\circ\{(z_{k-1}, q_{2k+1})\}\circ\{r_m\}_{m\in[2k-i]}$. Here 
                        
                        $\{(q_{j-1}, q_j)\}_{j\in [i-k-1]}$ is valid since $q_{b+1}\in \de(q_b, 0)=\de(q_b, 1)$ for $0\le b\le 2k$ from Case 2; 
                        
                        $\{(q_{i-k-1}, z_0)\}$ is valid due to $\de(q_a,0)=\{q_{a+1}, z_0\}$ for $0\le a\le k-1$ and $w_{i-k}=0$; 
                        
                        $\{(z_{l-1}, z_l)\}$ is valid due to $\de(z_c,0)=\de(z_c,1)=\{z_{c+1}\}$ for $0\le c\le k-2$;
                        
                        $\{(z_{k-1}, q_{2k+1})\}$ is valid due to $\de(z_{k-1},1)=\{q_{2k+1}\}$ and $w_i=1$ (the length of the walk before is $(i-k-1)+(1)+(k-1)=i-1$, thus we read the character $w_i$ during $\{(z_{k-1}, q_{2k+1})\}$);
                        
                        and finally $\{r_m\}_{m\in[2k-i]}$ is valid due to $\de(q_{2k+1},0)=\de(q_{2k+1},1)=\{q_{2k+1}\}$.

                        So, we have shown our walk is valid and reaches the final state $q_{2k+1}$. Note that all indexes are valid due to basic algebra and arithmetic manipulations.
                    }

                    For subsubcase (b) we have shown $w$ is accepted by $N_k$ thus $w\in L(N_k)$. And by definition of $X_k$, $w\in X_k$. So for subsubcase (b) $w\in L(N_k)$ if and only if $w\in X_k$.
                }

                For subcase (1) we have shown $w\in L(N_k)$ and $w\in X_k$. Thus $w\in L(N_k)$ if and only if $w\in X_k$.

                \subcase $\exists y\in\{0,1\}^k. (w=y\cd y)$.
                \indenv{
                    Since $\exists y\in\{0,1\}^k. (w=y\cd y)$, we instantiate such $y$.

                    By definition of $X_k$, $w\notin X_k$. We now show $w\notin L(N_k)$.

                    To this end, since $|w|=2k$, and the length of path to read $q_{d}$ where $d\in[2k-1]\cup\{0\}$ is exactly $d$ by our definitions of $\de$: $q_{d}$ can only be reached from $q_{d-1}$ for $d\in[2k-1]$ and $q_0$ can only be reached only by the string $\la$, solving the simple recurrence relation we have $q_{b}$ can only be reached by the strings with length $b$ for all $b\in[2k-1]\cup\{0\}$.

                    So, since $|w|=2k>2k-1$, final states $\{q_d\mid d\in[2k-1]\cup\{0\}\}$ cannot be reached by $w$, now we just have to show $q_{2k+1}$ cannot be reached by $w$.

                    \indenv{
                        To obtain a contradiction, assume there exists a walk that $q_{2k+1}$ is reached by $w$ from $q_0$.

                        There are 3 possible paths to reach $q_{2k+1}$: from $q_{2k}$, from $p_{k-1}$, and from $z_{k-1}$.

                        \tit{Subsubcase (a)} $w$ can be reached from $q_{2k}$.
                        \indenv{
                            For this case since to reach $q_{2k-1}$, the path must be at least length $2k-1$, and to reach $q_{2k}$, the path must be at least length $2k$, so the path must be at least length $2k+1$ to reach $q_{2k+1}$, which is a contradiction to $|w|=2k$.
                        }
                        For this subsubcase (a) contradiction occured.

                        \tit{Subsubcase (b)} $w$ can be reached from $p_{k-1}$.
                        \indenv{
                            Since $q_{2k+1}\in\de(p_{k-1}, 0)$ and $\de(p_{k-1},1)=\nil$, this implies $w_{i}=0$ for some $i\in[2k-1]\cup\{0\}$, since the shortest path to reach $p_{k-1}$ is length $k$ (from $q_0$ to $p_0$ and from $p_0$ to $p_{k-1}$), this gives $k+1\le i\le 2k-1$. Since $w=kk$ and $w_i=0$, this shows $w_{i-k}=0$ (valid index since $i\ge k+1$). However, since the path from $p_0$ to $p_{k-1}$ is exactly length $k$, and $p_0$ can only be reached by character $1$, this implies $w_{i-k}=1$, which is a contradiction.
                        }
                        For this subsubcase (b) contradiction occured.

                        \tit{Subsubcase (c)} $w$ can be reached from $z_{k-1}$.
                        \indenv{
                            Since $q_{2k+1}\in\de(z_{k-1}, 1)$ and $\de(z_{k-1},0)=\nil$, this implies $w_{i}=1$ for some $i\in[2k-1]\cup\{0\}$, since the shortest path to reach $z_{k-1}$ is length $k$ (from $q_0$ to $z_0$ and from $z_0$ to $z_{k-1}$), this gives $k+1\le i\le 2k-1$. Since $w=kk$ and $w_i=1$, this shows $w_{i-k}=1$. However, since the path from $z_0$ to $z_{k-1}$ is exactly length $k$, and $z_0$ can only be reached by character $0$, this implies $w_{i-k}=0$, which is a contradiction.
                        }
                        For this subsubcase (c) contradiction occured.
                    }
                    For all cases contradiction occured, so $q_{2k+1}$ cannot be reached by $w$. Since $w$ cannot reach all final states, we conclude $w\notin L(N_k)$, and by definition of $X_k$, $w\notin X_k$. Therefore $w\in L(N_k)$ if and only if $w\in X_k$.
                }
                
                For subcase (2) we have shown $w\notin L(N_k)$ and $w\notin X_k$. Thus $w\in L(N_k)$ if and only if $w\in X_k$.
            }

            We conclude for Case 3 $w\in L(N_k)$ if and only if $w\in X_k$.

        \end{proofcases}

        For all cases we have shown $w\in L(N_k)$ if and only if $w\in X_k$.
    }

    Since $w$ is arbitrary, we conclude $\forall w\in\{0,1\}^*. w\in L(N_k)$ if and only if $w\in X_k$. Since $L(N_k)\subseteq \{0,1\}^*$ and $X_k\subseteq \{0,1\}^*$, by definition of set equality we conclude $L(N_k)=X_k$.
}

Since $k$ is arbitrary, we therefore have shown that $L(N_k)=X_k$ for all $k\in\Z^+$. \qed




\end{document}