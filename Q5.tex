\documentclass[11pt]{article}
\usepackage{fullpage}
\usepackage{comment}
\usepackage{amsmath}
\usepackage{amssymb}
\usepackage{amsthm}
\setlength{\parindent}{0pt}
\setlength{\parskip}{0.3cm}
\newcommand{\nats}{\mathbb{N}}
\newcommand{\ints}{\mathbb{Z}}

%%% Logic
\newcommand{\nnot}{\mathrm{NOT}}
\newcommand{\aand}{\,\,\mathrm{AND}\,\,}
\newcommand{\oor}{\,\,\mathrm{OR}\,\,}
\newcommand{\iimplies}{\,\,\mathrm{IMPLIES}\,\,}
\newcommand{\xor}{\,\,\mathrm{XOR}\,\,}
\newcommand{\iif}{\,\,\mathrm{IFF}\,\,}

\begin{document}
\begin{center}
{\bf \Large \bf CSC240 Winter 2024 Homework Quiz 5}\\
due February 9, 2024
\end{center}

A propositional formula $f$ is in \textit{negation normal form} if it is built from literals using only conjunction and disjunction.
For instance, all CNF and DNF formulas are in negation normal form.

Give a recursive definition for the set of propositional formulas in negation normal form.\\

Let $N$ = set of propositional formulas in negation normal form.

Base Case: Set of all literals $\subseteq N$.

Constructor Case: If $f_1, f_2 \in N$, then $(f_1 \aand f_2) \in N$, and $(f_1 \oor f_2) \in N$.


\end{document}
