\documentclass[11pt]{article}
\usepackage{fullpage}
\usepackage{comment}
\usepackage{amsmath}
\usepackage{amssymb}
\usepackage{amsthm}
\usepackage{changepage}
\setlength{\parindent}{0pt}
\setlength{\parskip}{0.3cm}
\newcommand{\nats}{\mathbb{N}}
\newcommand{\ints}{\mathbb{Z}}

%%% Logic
\newcommand{\nnot}{\mathrm{NOT}}
\newcommand{\aand}{\,\,\mathrm{AND}\,\,}
\newcommand{\oor}{\,\,\mathrm{OR}\,\,}
\newcommand{\iimplies}{\,\,\mathrm{IMPLIES}\,\,}
\newcommand{\xor}{\,\,\mathrm{XOR}\,\,}
\newcommand{\iif}{\,\,\mathrm{IFF}\,\,}
\newcommand{\explain}[1]{{\color{blue} #1}}

\begin{document}
\begin{center}
{\bf \Large \bf CSC240 Winter 2024  Quiz 7}\\
due March 15, 2024
\end{center}

Consider the following algorithm DIV that, given positive integers $ a $ and $ b$,
computes the quotient $ q $ and remainder $ r $ of $ a $ divided by $ b$:
\begin{center}
\begin{tabbing}
DIV\=($a,b$):\\
1\>$ p \leftarrow 1 $\\
2\>$ s \leftarrow b $\\
3\>while \=$ s\leq a $ do\\
4\>\>$ s \leftarrow 2 \times s $\\
5\>\>$ p \leftarrow 2\times p $\\
6\>$ q \leftarrow 0 $\\
7\>$ r \leftarrow a $\\
8\>while $ s\neq b $ do\\
9\>\>$ s \leftarrow s $~div~2\\
10\>\>$ p \leftarrow p $~div~2\\
11\>\>if \=$ r\geq s $~then\\
12\>\>\>$ r \leftarrow r-s $\\
13\>\>\>$ q \leftarrow q+p$
\end{tabbing}
\end{center}
\begin{enumerate}
\item Write a precise statement of what it means for this algorithm to be correct.

%%%%%%%%% Your solution begins here %%%%%%%%%%
{\bf Solution.}

The algorithm is correct (totally correct) if it is both partially correct and terminates.

{\bf Precondition:} $a$ and $b$ are positive integers.

{\bf Postcondition:} $q$ and $r$ are integers such that $a=qb+r$ and $0\leq r<b$. Also, $a$ and $b$ are not changed. 

Moreover, for the algorithm to be partially correct, we need the postcondition to be true whenever the algorithm terminates.\\



\item Write a loop invariant for both while loops that relates $s$ and $p$. You do not have to prove that
it is a loop invariant.

%%%%%%%%% Your solution begins here %%%%%%%%%%
{\bf Solution.}

{\bf Loop Invariant for both while loops:} $s=p\times b$.\\


\item Prove that $ a=qb+r $ is a loop invariant for the second while loop.
%%%%%%%%% Your solution begins here %%%%%%%%%%
{\bf Solution.}

\begin{proof}
    Initially from lines 6 and 7 we can see that $a=0\times b + a = q\times b + r$, the loop invariant holds.

    Consider an arbitrary iteration of the loop,
    \begin{adjustwidth}{1cm}{}
        Let $q', r'$ be the values of $q$ and $r$ at the beginning of the iteration, and let $q'', r''$ be the values of $q$ and $r$ at the end of the iteration.

        Suppose the invairant is true at the beginning of the iteration, on line 12 and 13, since by question 2 we have $s=p\times b$, substitute $p\times b$ into line 12's $s$ and we will see that $r''=r' - p\times b$ and on line 13 $q''=q'+p$.

        Combine together we have $q''b+r'' = (q'+p)b + (r'-p\times b) = q'b + r'=a$, the loop invariant holds.
        
    \end{adjustwidth}

    Therefore by induction we have that $a=qb+r$ is a loop invariant for the second while loop.
    
\end{proof}


\end{enumerate}
\end{document}
