\documentclass[12pt]{article}
\usepackage{fullpage}
\usepackage{amssymb,amsmath}
%amsthm,amsfonts}

\setlength{\parskip}{0.3cm}
\setlength{\parindent}{0cm}

%%% Logic
\newcommand{\nnot}{\mathrm{NOT}}
\newcommand{\aand}{\,\,\mathrm{AND}\,\,}
\newcommand{\oor}{\,\,\mathrm{OR}\,\,}
\newcommand{\iimplies}{\,\,\mathrm{IMPLIES}\,\,}
\newcommand{\xor}{\,\,\mathrm{XOR}\,\,}
\newcommand{\iif}{\,\,\mathrm{IFF}\,\,}
\newcommand{\nand}{\,\,\mathrm{NAND}\,\,}

%%% Sets
\newcommand{\nats}{\mathbb{N}}
\newcommand{\ints}{\mathbb{Z}}

\begin{document}
\begin{center}
{\Large CSC 240 Winter 2024 Homework Assignment 1}\\
due January 18, 2024
\end{center}

My name and student number: Joseph Siu, 1010085701\\
The list of people with whom I discussed this homework assignment:

Sanchit Manchanda

Serif Wu

Hrithik Parag Shah

Abhi Prajapati

Sepehr Jafari


\begin{enumerate}
\item
Let $U$ be the set of all people and let $\mathcal{P}(U)$ be the set of all subsets of $U$.\\
Let $In: U \times \mathcal{P}(U)  \rightarrow \{$T,F\} be the predicate such that 
$In(x,G) = $ T when person $x$ is in the set of people $G$.\\
Let $size:\mathcal{P}(U) \rightarrow \nats$ be the function 
such that $size(G)$ is the number of people in the set $G$.\\
Let $K: U \times U \to \{$T,F\} be the predicate such that 
$K(a,b) =$ T when person $a$ and person $b$ know each other.\\
You may assume that $K$ is symmetric, i.e.~$\forall a \in U. \forall b \in U. [K(a,b) \iif  K(b,a)]$.

\begin{enumerate}
\item
Consider the following predicate about a set of people $G \in \mathcal{P}(U)$:
\begin{align*}
\exists H \in \mathcal{P}(U). & \forall x \in H. \forall y \in H.\\
& [(size(H)= 40)  \aand  (\nnot(K(x,y)) \oor (x=y)) \aand In(x , G)].
 \end{align*}
Express this predicate using at most 15 English words. The letter $G$ counts as one word.
Justify your answer. 
    
\vspace{.25in}\textbf{Solution}\vspace{.10in}\\

The predicate is equivalent to ``The set G contains 40 people who don't know each other."\\

First we can see the predicate (beside the quantification parts) is mainly formed by 2 AND connections. The first part is "A set of people H has 40 people", the second part is "no one in H knows each other", and the third part is "all people in H are also in G" which means H is a subset of G. Combining these 3 parts with the 3 quantifications, we obtained our sentence. 

\item
Consider the following sentence in predicate logic.
\begin{align*}
\exists n \in \mathbb{N}. & \forall G \in \mathcal{P}(U). \exists H \in \mathcal{P}(U). \forall x \in H. \forall y \in H.\\
& [((size(H)= 40) \oor  (size(G)  < n)) \aand \\ & (\nnot(K(x,y)) \oor (x=y) \oor (size(G)  < n)) \aand \\ & (In(x ,G) \oor  (size(G)  < n))].
\end{align*}
Express this sentence using at most 20 English words.
Justify your answer. 
 
\vspace{.25in}\textbf{Solution}\vspace{.10in}\\

The predicate is equivalent to "Any group of people either has fewer than a fixed number of people, or contains 40 strangers to each other."
\\
 First, the predicate is logically equivalent to 
\begin{align*}
\exists n \in \mathbb{N}. & \forall G \in \mathcal{P}(U). \exists H \in \mathcal{P}(U). \forall x \in H. \forall y \in H.\\
& (size(G)  < n) \oor [(size(H) = 40) \aand \\ & (\nnot(K(x,y)) \oor (x=y)) \aand \\ & (In(x ,G))].
\end{align*}

Now we move $\exists H \in \mathcal{P}(U). \forall x \in H. \forall y \in H.$ within the second part (right side) of the $\oor$ connector, this is allowed becasue these quantifications have no affect on the first (left side) part,

\begin{align*}
    \exists n \in \mathbb{N}. & \forall G \in \mathcal{P}(U). (size(G)  < n) \oor \\ & [\exists H \in \mathcal{P}(U). \forall x \in H. \forall y \in H. (size(H) = 40) \aand  \\ & (\nnot(K(x,y)) \oor (x=y)) \aand (In(x ,G))].
\end{align*}

We can now see the second (right side) part is precisely our Part A of this question. So, the first two quantifications give "there is a natural number n, for any sets of people G", then the first part of $\oor$ gives "the size of the set G is less than the natural number n", and the second part of $\oor$ gives "there is a set of 40 people who don't know each other and are all in the set of people G". Combining these 3 parts, we obtained our sentence, completing the justification. 
 

\end{enumerate}



\item
A graph consists of a set of vertices  $V$ and a set of edges between pairs of vertices.\\
The set of edges can be represented by a symmetric, irreflexive predicate $e: V \times V \to \{T,F\}$, where $e(a,b) = T$ denotes that there is an edge between vertex $a$ and $b$.\\
Symmetric means  $\forall a \in V. \forall b \in V. (e(a,b) \iif e(b,a))$ and\\
irreflexive means  $\forall a \in V.\nnot (e(a,a))$).

We say that two vertices $a$ and $b$ are \emph{adjacent} if there is an edge between them.\\
The \emph{degree of a vertex}  is the number of vertices that are adjacent to it.\\
The \emph{maximum degree of a graph} is the maximum of the degrees of its vertices.

A \emph{colouring} of the graph is a function from $V$ to a set of colours. We will use the natural numbers
 as the set of colours. Then $\nats^V$ denotes the set of all colourings of $V$.

A \textit{proper colouring} is a \textit{colouring} such that no two adjacent vertices have the same colour.


Using predicate logic, express the following statement  about a graph with vertex set $V$ and whose set of edges is represented by the predicate $e$:\\
\ \ \ \ \ ``If the graph has maximum degree two, then it can be properly coloured using at most three colours.''\\
You cannot use any constants. 
The only predicate (besides $e$) you can use is binary equality, $=$.

To get full marks, you must use $\iimplies$ at most once and $\nnot$ at most four times.

Use parentheses and brackets when necessary to avoid ambiguity.\\
Justify your answer  by briefly explaining what its various parts mean.

\vspace{.25in}\textbf{Solution}\vspace{.10in}

The statement is logically equivalent to 

\begin{multline*}
    [[\forall a\in V.\forall b\in V.\forall c\in V.\forall d\in V.\\
    \nnot(e(a,b)\aand e(a,c)\aand e(a,d))\oor((b=c)\oor(b=d)\oor(c=d))]\\
    \aand[\exists g\in V.\exists h\in V. \exists i\in V. e(g,h)\aand e(g,i)\aand (\nnot(h=i))]]\\
    \iimplies[\exists f\in\nats^V.\forall w\in V.\forall x\in V.\forall y\in V.\forall z\in V.\\
    [(f(w)=f(x))\oor(f(w)=f(y))\oor(f(w)=f(z))] \\
    \aand [\nnot(e(w,x)\aand(f(w)=f(x)))]].
\end{multline*} 
\\
The part before IMPLIES has two parts connected by $\aand$. First, $$\forall a\in V.\forall b\in V.\forall c\in V.\forall d\in V.$$$$\nnot(e(a,b)\aand e(a,c)\aand e(a,d))\oor((b=c)\oor(b=d)\oor(c=d))$$ is logically equivalent to $$\forall a\in V.\forall b\in V.\forall c\in V.\forall d\in V.$$$$(e(a,b)\aand e(a,c)\aand e(a,d))\iimplies((b=c)\oor(b=d)\oor(c=d))$$, this means if a vertex is adjacent to 3 other vertices (may repeat the same vertex), then at least two of these 3 vertices are the same vertex. What this says is that the maximum degree is at most 2. Second, $$\exists g\in V.\exists h\in V. \exists i\in V. e(g,h)\aand e(g,i)\aand (\nnot(h=i))$$ means there is a vertex adjacent to 2 distinct vertices, this gives the maximum degree of the graph is at least 2. Combining these 2 we have the maximum degree of the graph is precisely 2.

The part after IMPLIES also contains two main parts connected by $\aand$. First, $$\exists f\in\nats^V.\forall w\in V.\forall x\in V.\forall y\in V.\forall z\in V.(f(w)=f(x))\oor(f(w)=f(y))\oor(f(w)=f(z))$$ means there exists a colouring function which, for any 4 vertices, at least 2 of them must have the same colour, this gives a colouring function of the graph with at most 3 colours in the range of the function. Second,  $$\nnot(e(w,x)\aand(f(w)=f(x)))$$ gives it cannot be true that two vertices are adjacent and have the same colour, by definition this shows the function $f$ is a proper colouring function. Combining with the first part, the graph needs to have a proper colouring function which only has at most 3 colours in its range.

Therefore, as combining everything before and after $\iimplies$, we conclude our predicate is precisely the sentence required.

\end{enumerate}
\end{document}
